%%%%%%%%%%%%%%%%%%%%%%%%%%%%%%%%%%%%%%%%%%%%%%%%%%%%%%%%%%%%%%%%%%%%%%%%%%%%%%%
%%%%%%%%%%%%%%%%%%%%%%%%%%%%%%%%%%%%%%%%%%%%%%%%%%%%%%%%%%%%%%%%%%%%%%%%%%%%%%%
\chapter{\emph{Consort}: a model of composition}
\label{chap:a-model-of-composition}
%%%%%%%%%%%%%%%%%%%%%%%%%%%%%%%%%%%%%%%%%%%%%%%%%%%%%%%%%%%%%%%%%%%%%%%%%%%%%%%
%%%%%%%%%%%%%%%%%%%%%%%%%%%%%%%%%%%%%%%%%%%%%%%%%%%%%%%%%%%%%%%%%%%%%%%%%%%%%%%

\begin{comment}
<abjad>[hide=true]
import collections
import consort
</abjad>
\end{comment}

\begin{comment}
\begin{markdown}
#   Specification & Interpretation
-   Materials
-   Configuration
-   Templating
-   Layers
-   Coarse to fine
-   Rhythm first
-   What is specification? What is a specifier? What is configuration and
    aggregation?
-   What should happen musically, where should it happen?
-   What is material?
-   What is music?
-   Rhythm is interpreted first, as all other parameters depend on it.
-   Rhythm is interpreted from "coarse" to "fine": from the level of phrase
    boundaries to the level of individual notes, rests, tuplets and ties.
-   This discussion only focuses on notation, nothing related to aesthetic
    experience, physical modeling or anything else. This is a tool for a
    specific composer to create scores, not a discussion explicitly of why they
    would work this way (although that should be discussed in the conclusion).
-   Specification and interpretation conceive of the score as a single, hugely
    complex expression.
-   Templating as variation.
-   Define what composition means here: laying out symbols on the page.
-   This way of thinking and working does not attempt to define or even model
    concepts like "melody" or even "phrase". They're too vague. If we use that
    terminology at all, it is only in the most incredibly constrained way.
\end{markdown}
\end{comment}

\begin{markdown}

Consort, a Python library written as an extension to Abjad, models the
composition of notated musical score as a repeated cycle containing three
distinct stages: *specification*, *interpretation* and *visualization*.

*Specification* describes how *out-of-time materials* -- both concrete and
programmatic -- should be deployed *in-time* in a *segment* of musical score as
notation. Materials encompass abstractions -- such as pitch sets or collections
of performance technique indications --, concrete fragments of *music* --
narrowly defined here as any contiguous selection of score components --, and
procedures for producing, altering or embellishing music such as rhythm-makers
or attachment-handlers. Score segments comprise any contiguous passage of
music, demarcating an area of compositional concern. Consort treats scores as
comprised of at least one segment, but potentially many more concatenated
together. Any segment may of course contain arbitrarily complex inner
structuring. Separation of scores into distinct segments acts then mainly as an
aid for the composer, both by simplifying the complexity of the current
specification under consideration, and by allowing the typesetting engine --
LilyPond -- to display more manageable amounts of notation than the full score,
thus speeding up the cycle of specifying, interpreting and visualizing.

Composers specify segments by creating and progressively configuring
*segment-makers*, classes which conceptually mirror the rhythm- and
timespan-makers described in \autoref{chap:time-tools}, but on a much larger
scale. Such configuration parameters include score template, tempo, permitted
meters, and desired duration. Most importantly, segment-makers may be
configured with any number of *music settings*, which aggregate a
timespan-maker, a target timespan and any number of *music specifiers* --
bundles of materials which describe how phrases of music should be produced
within a single voice. Music settings object-model both *when* and in *which*
voices musical materials should be deployed. The order in which composers
configure segment-makers with music settings defines each music setting's
*layer*, determining how overlapping events in a single voice will mask one
another. Score materials, including music settings, music specifiers,
timespan-makers and any other class pertinent to score creation -- potentially
even other segment-makers, may be defined from scratch in the same code module
as the segment-maker currently being configured, templated from another
material, or simply imported into the segment definition's namespace.

At any point during specification, a segment-maker may be interpreted to
produce an illustration. Score interpretation proceeds conceptually much like
compilation in classical computing, where a compiler parses an instruction set
written in some source language into an intermediate representation and then
transforms that same intermediate representation into instructions executable
on a target platform. In Consort's interpretation stage, the compiler is the
segment-maker itself, and the source instruction set its configuration -- its
tempo, permitted meters, music settings and so forth. Timespan inventories
produced by each music setting's timespan-maker, populated with timespans
annotated with music specifiers perform the role of the intermediate
representation. This intermediate representation acts as a *maquette*, blocking
out where in the resulting score segment various materials should be deployed.
The target of score interpretation is, unsurprisingly, a fully-fledge score
aggregated from Abjad score components.

Interpretation itself takes place in two broad stages -- rhythmic
interpretation, followed by non-rhythmic interpretation -- with the first stage
producing a score populated solely with rhythmic information, and the second
stage applying grace notes, pitches, indicators, spanners and various
typographic overrides to the previously-constructed rhythmic skeleton.

<describe very high level, then lower>

Before either rhythmic or non-rhythmic interpretation, the segment-maker first
creates a *segment-session*, an object which keeps track of the products of
interpretation -- the interpreted score, but also generated meters,
attack-points and timespans -- for the segment-maker, allowing the
segment-maker itself to remain stateless while it interprets.

During rhythmic interpretation, the segment-maker populates an inventory of
timespans by calling each of its music-settings. 

Music settings with *independent* timespan-makers -- those which do not depend
on the contents of a previously created timespan inventory -- are called in a
first pass, and those with *dependent* timespan-makers second.

When populating independent timespans, each music settings is called in turn to
produce timespans according to their configured timespan-makers,
*timespan-identifiers* -- optional specifications of which portion of the
segment's overall timespan to operate within -- and voice-associated music
specifiers.

Music setting must *resolve* both their timespan identifier -- in the case of
procedural identifiers such as ratio-parts expressions which may not even know
their target timespans during specifications -- and their voice-name
abbreviations against the segment's desired duration and score template
respectively.

This resolution results in ... <x products>

-   Mention composite music specifiers and music specifier sequences
-   Also timespan quantization

Once resolved, each music setting can call its timespan-maker to create
timespans with the appropriate music specifiers, target timespans and layer,
adding the resulting inventory to the growing collection of performed and
silent timespans produced by previous music settings

-   timespan creation actually takes place not directly from timespan-makers
    but via music specifier sequences (composite music specifiers aggregate
    these together)

-   populating independent timespans:
    -   populate demultiplexed timespans
    -   Find meters
    -   Find meter boundaries
    -   Demultiplex 
        -   resolve cascading overlap across all timespans
    -   Split timespans at meter boundaries, discarding overly short timespans
    -   Discard malformed timespans (as a precaution)
    -   Consolidate timespans, discarding if below minimum phrase duration,
        populating the `divisions` property of the result
    -   Inscribe the consolidated timespans
        -   get rhythm maker (under what circumstances?)
        -   make simple music
        -   consolidate rests
        -   attach beams and music specifiers (with scope) to each consolidated
            container
        -   reconfigure the timespan with the created music, but without the
            divisions
    -   Prune meters

-   populating dependent timespans
    -   this occurs almost identically as with independent timespans,
        but without any meter finding
    -   populate demultiplexed timespans
    -   demultiplex
    -   split at meter boundaries
    -   consolidate
    -   inscribe

-   populate silent timespans
    -   creates timespans, splits and inscribes in one pass
    -   there are no more layers to create, so no need to resolve overlap
        conflicts

-   rewrite meters (out of score, for efficiency)
-   populate inscribed timespans music into the score
-   collect attack points

During non-rhythmic interpretation:

-   Non-rhythmic interpretation
    -   Grace handling
    -   Pitch handling
    -   Attachment handling
    -   Post-processing

Once interpreted, a segment-maker's illustration may be persisted to disk as
LilyPond syntax for inclusion in other LilyPond files, rendered as a PDF for
viewing, or even serialized for other purposes. Composers study the results of
interpretation, make changes to each segment's specification, and re-interpret
as necessary, a large-scale re-enactment of interactive programming's pervasive
*read-eval-print* loop paradigm.

What follows is a detailed analysis of the various algorithms and subroutines
employed during specification and interpretation.

\end{markdown}

\section{Segment makers}

\begin{markdown}
#   Segment-makers
-   Permitted time signatures
-   Tempo
-   Desired duration
-   Score template
    -   LilyPond will automatically concatenate scores with identical
        context hierarchies. All contexts need to be present in every
        concatenated segment, otherwise LilyPond will concatenate
        incorrectly. However, we can use various typographic overrides to
        make it appear that a context has disappeared.
    -   Consort's ScoreTemplateManager helps create regularized score
        templates.
-   Music settings
\end{markdown}

\subsection{Segment-maker examples}

\begin{comment}
<abjad>
segment_maker = consort.SegmentMaker(
    desired_duration_in_seconds=9,
    omit_stylesheets=True,
    permitted_time_signatures=[(3, 4)],
    score_template=templatetools.GroupedRhythmicStavesScoreTemplate(
        staff_count=2,
        with_clefs=True,
        ),
    tempo=indicatortools.Tempo((1, 4), 60),
    )
</abjad>
\end{comment}

%%% ABJADBOOK START %%%
\begin{singlespacing}
\vspace{-0.5\baselineskip}
\begin{lstlisting}
>>> segment_maker = consort.SegmentMaker(
...     desired_duration_in_seconds=9,
...     omit_stylesheets=True,
...     permitted_time_signatures=[(3, 4)],
...     score_template=templatetools.GroupedRhythmicStavesScoreTemplate(
...         staff_count=2,
...         with_clefs=True,
...         ),
...     tempo=indicatortools.Tempo((1, 4), 60),
...     )
\end{lstlisting}
\end{singlespacing}
%%% ABJADBOOK END %%%

\begin{comment}
<abjad>[stylesheet=../consort.ily]
show(segment_maker, verbose=False)
</abjad>
\end{comment}

%%% ABJADBOOK START %%%
\begin{singlespacing}
\vspace{-0.5\baselineskip}
\begin{lstlisting}
>>> show(segment_maker, verbose=False)
\end{lstlisting}
\noindent\includegraphics[max width=\textwidth,]{assets/lilypond-5551103a6156a0c950aaf871d1206d96.pdf}
\end{singlespacing}
%%% ABJADBOOK END %%%

\begin{comment}
<abjad>[stylesheet=../consort.ily]
faster_segment_maker = new(
    segment_maker,
    tempo=indicatortools.Tempo((1, 4), 120),
    )
show(faster_segment_maker, verbose=False)
</abjad>
\end{comment}

%%% ABJADBOOK START %%%
\begin{singlespacing}
\vspace{-0.5\baselineskip}
\begin{lstlisting}
>>> faster_segment_maker = new(
...     segment_maker,
...     tempo=indicatortools.Tempo((1, 4), 120),
...     )
>>> show(faster_segment_maker, verbose=False)
\end{lstlisting}
\noindent\includegraphics[max width=\textwidth,]{assets/lilypond-88bcdb27b1ad17661a645e4df9a41e96.pdf}
\end{singlespacing}
%%% ABJADBOOK END %%%

\section{Music settings}

\begin{markdown}
#   Music settings
-   What kind of music, laid out in what pattern, in which general
    time frame, for which voices, in what layer?
-   A timespan-maker
-   A target timespan
    -   A timespan.
    -   A timespan inventory.
    -   A ratio/parts expression.
        -   Demonstrate. Check.
-   Voice abbreviation / music specifier pairs
    -   Why abbreviate? It's because of Python's key=value syntax for
        keyword arguments.
    -   How do we convert abbreviation to an actual context name?
        -   Score templates need to define a mapping.
        -   Demonstrate. Check.
    -   What is the context name for?
        -   Name-wise indexing in an actual score.
        -   Demonstrate. Check.
    -   Also, composite music specifiers: when two contexts need to be
        discussed as a single unit.
        -   Composite music specifiers need to convert one abbreviation into
            two more abbreviations, each mapped to an actual context name.
        -   Consort's ScoreTemplateManager takes care of calculating and
            caching the appropriate names and abbreviations.
-   Layer is implicit, derived from the order of setting definitions.
    -   Recall the discussion of timespan layer from earlier chapters.
\end{markdown}

\subsection{Ratio-parts expression examples}

\begin{comment}
<abjad>
timespan = timespantools.Timespan(0, 8)
show(timespan)
ratio_parts_expression = consort.RatioPartsExpression(
    ratio=[1, 2, 1],
    parts=[0, 2],
    )
result = ratio_parts_expression(timespan)
show(result)
</abjad>
\end{comment}

%%% ABJADBOOK START %%%
\begin{singlespacing}
\vspace{-0.5\baselineskip}
\begin{lstlisting}
>>> timespan = timespantools.Timespan(0, 8)
>>> show(timespan)
\end{lstlisting}
\noindent\includegraphics[max width=\textwidth,]{assets/lilypond-a34adb1cfbda637e38739ddd84494442.pdf}
\begin{lstlisting}
>>> ratio_parts_expression = consort.RatioPartsExpression(
...     ratio=[1, 2, 1],
...     parts=[0, 2],
...     )
>>> result = ratio_parts_expression(timespan)
>>> show(result)
\end{lstlisting}
\noindent\includegraphics[max width=\textwidth,]{assets/lilypond-53d850fdf03d8308e4c1a9b0c39b239a.pdf}
\end{singlespacing}
%%% ABJADBOOK END %%%

\begin{comment}
<abjad>
ratio_parts_expression = new(
    ratio_parts_expression,
    timespan=timespantools.Timespan(
        start_offset=Offset(1, 4),
        ),
    )
result = ratio_parts_expression(timespan)
show(result, range_=(0, 8))
</abjad>
\end{comment}

%%% ABJADBOOK START %%%
\begin{singlespacing}
\vspace{-0.5\baselineskip}
\begin{lstlisting}
>>> ratio_parts_expression = new(
...     ratio_parts_expression,
...     timespan=timespantools.Timespan(
...         start_offset=Offset(1, 4),
...         ),
...     )
>>> result = ratio_parts_expression(timespan)
>>> show(result, range_=(0, 8))
\end{lstlisting}
\noindent\includegraphics[max width=\textwidth,]{assets/lilypond-79f16a4a8418396c8645b819bd11c617.pdf}
\end{singlespacing}
%%% ABJADBOOK END %%%

\subsection{Score template and abbreviation examples}

\begin{comment}
<abjad>
import armilla
armilla_score_template = armilla.makers.ArmillaScoreTemplate()
armilla_score = armilla_score_template()
print(format(armilla_score))
for item in armilla_score_template.context_name_abbreviations.items():
    abbreviation, context_name = item
    print(abbreviation, context_name)

for item in armilla_score_template.composite_context_pairs.items():
    grouping_abbreviation, grouped_abbreviations = item
    print(grouping_abbreviation, grouped_abbreviations)

</abjad>
\end{comment}

%%% ABJADBOOK START %%%
\begin{singlespacing}
\vspace{-0.5\baselineskip}
\begin{lstlisting}
>>> import armilla
>>> armilla_score_template = armilla.makers.ArmillaScoreTemplate()
>>> armilla_score = armilla_score_template()
>>> print(format(armilla_score))
\context Score = "Armilla Score" <<
    \tag #'time
    \context TimeSignatureContext = "TimeSignatureContext" {
    }
    \tag #'viola-1
    \context StringPerformerGroup = "Viola 1 Performer Group" \with {
        instrumentName = \markup {
            \hcenter-in
                #10
                "Viola 1"
            }
        shortInstrumentName = \markup {
            \hcenter-in
                #10
                "Va. 1"
            }
    } <<
        \context BowingStaff = "Viola 1 Bowing Staff" {
            \clef "percussion"
            \context Voice = "Viola 1 Bowing Voice" {
            }
        }
        \context FingeringStaff = "Viola 1 Fingering Staff" {
            \clef "alto"
            \context Voice = "Viola 1 Fingering Voice" {
            }
        }
    >>
    \tag #'viola-2
    \context StringPerformerGroup = "Viola 2 Performer Group" \with {
        instrumentName = \markup {
            \hcenter-in
                #10
                "Viola 2"
            }
        shortInstrumentName = \markup {
            \hcenter-in
                #10
                "Va. 2"
            }
    } <<
        \context BowingStaff = "Viola 2 Bowing Staff" {
            \clef "percussion"
            \context Voice = "Viola 2 Bowing Voice" {
            }
        }
        \context FingeringStaff = "Viola 2 Fingering Staff" {
            \clef "alto"
            \context Voice = "Viola 2 Fingering Voice" {
            }
        }
    >>
>>
\end{lstlisting}
\begin{lstlisting}
>>> for item in armilla_score_template.context_name_abbreviations.items():
...     abbreviation, context_name = item
...     print(abbreviation, context_name)
...
('viola_1', 'Viola 1 Performer Group')
('viola_1_rh', 'Viola 1 Bowing Voice')
('viola_1_lh', 'Viola 1 Fingering Voice')
('viola_2', 'Viola 2 Performer Group')
('viola_2_rh', 'Viola 2 Bowing Voice')
('viola_2_lh', 'Viola 2 Fingering Voice')
\end{lstlisting}
\begin{lstlisting}
>>> for item in armilla_score_template.composite_context_pairs.items():
...     grouping_abbreviation, grouped_abbreviations = item
...     print(grouping_abbreviation, grouped_abbreviations)
...
('viola_1', ('viola_1_rh', 'viola_1_lh'))
('viola_2', ('viola_2_rh', 'viola_2_lh'))
\end{lstlisting}
\end{singlespacing}
%%% ABJADBOOK END %%%

\begin{comment}
<abjad>
armilla_score['Viola 1 Fingering Voice']
</abjad>
\end{comment}

%%% ABJADBOOK START %%%
\begin{singlespacing}
\vspace{-0.5\baselineskip}
\begin{lstlisting}
>>> armilla_score['Viola 1 Fingering Voice']
Voice()
\end{lstlisting}
\end{singlespacing}
%%% ABJADBOOK END %%%

\section{Music specifiers}

\begin{markdown}
-   A bundle of descriptors for what kind of music should fill a series of
    1 or more timespans.
-   All of the descriptors are optional.
-   Rhythm-makers:
    -   Creates rhythms in the divisions defined by a series of contiguous
        timespans.
-   Grace-handlers:
    -   Adds grace notes to the start of logical ties in a patterned way.
    -   Processes the score time-wise by logical tie.
-   Pitch-handlers:
    -   Applies pitches to logical ties in a patterned way.
    -   Applies pitches to grace notes associated with a logical tie.
    -   Applies logical-tie-expressions which can convert logical ties from
        notes into chords, key-clusters or harmonics.
    -   Processes the score time-wise by logical tie.
        -   The goal is to limit pitch class repetition both vertically and
            horizontally, but *only* with regard to phrases in the score scoped
            by each music specifier. Other phrases are not considered.
    -   Application rate: by logical tie, division, phrase
        -   This requires the SeedSession class for keeping track of many
            seeds, each advancing at a potentially different rate.
        -   This also requires the AttackPointSignature class, which caches
            information about each logical tie's position in its parent
            division, phrase and overall segment.
    -   MusicSpecifier: pitches are non-semantic (is this even used?)
    -   Maps different patterns of pitches and different patterns of operations
        across the timeline.
        -   PitchHandler: `get_pitch_choice_timespans()`
        -   Demonstrate.
    -   Can act on absolute pitches, or registered pitch classes.
    -   Other formulations are possible: selecting from vertical sonorities
        based on register curves. (This is not currently implemented, but maybe
        for Ersilia.)
    -   Demonstrate simple PitchHandler examples.
\end{markdown}

\subsection{Music specifier label examples}

\begin{comment}
<abjad>
unlabeled_music_specifier = consort.MusicSpecifier()
labeled_music_specifier = consort.MusicSpecifier(labels=['labeled'])
timespan_inventory = timespantools.TimespanInventory([
    consort.PerformedTimespan(
        layer=1,
        start_offset=0,
        stop_offset=4,
        music_specifier=labeled_music_specifier,
        voice_name='Voice 1',
        ),
    consort.PerformedTimespan(
        layer=1,
        start_offset=2,
        stop_offset=7,
        music_specifier=labeled_music_specifier,
        voice_name='Voice 2',
        ),
    consort.PerformedTimespan(
        layer=2,
        start_offset=6,
        stop_offset=8,
        music_specifier=unlabeled_music_specifier,
        voice_name='Voice 1',
        ),
    consort.PerformedTimespan(
        layer=2,
        start_offset=10,
        stop_offset=(25, 2),
        music_specifier=unlabeled_music_specifier,
        voice_name='Voice 2',
        ),
    consort.PerformedTimespan(
        layer=1,
        start_offset=11,
        stop_offset=14,
        music_specifier=labeled_music_specifier,
        voice_name='Voice 1',
        ),
    consort.PerformedTimespan(
        layer=1,
        start_offset=14,
        stop_offset=16,
        music_specifier=labeled_music_specifier,
        voice_name='Voice 1',
        ),
    consort.PerformedTimespan(
        layer=1,
        start_offset=15,
        stop_offset=16,
        music_specifier=labeled_music_specifier,
        voice_name='Voice 2',
        ),
    ])
show(timespan_inventory, key='voice_name')
</abjad>
\end{comment}

%%% ABJADBOOK START %%%
\begin{singlespacing}
\vspace{-0.5\baselineskip}
\begin{lstlisting}
>>> unlabeled_music_specifier = consort.MusicSpecifier()
>>> labeled_music_specifier = consort.MusicSpecifier(labels=['labeled'])
>>> timespan_inventory = timespantools.TimespanInventory([
...     consort.PerformedTimespan(
...         layer=1,
...         start_offset=0,
...         stop_offset=4,
...         music_specifier=labeled_music_specifier,
...         voice_name='Voice 1',
...         ),
...     consort.PerformedTimespan(
...         layer=1,
...         start_offset=2,
...         stop_offset=7,
...         music_specifier=labeled_music_specifier,
...         voice_name='Voice 2',
...         ),
...     consort.PerformedTimespan(
...         layer=2,
...         start_offset=6,
...         stop_offset=8,
...         music_specifier=unlabeled_music_specifier,
...         voice_name='Voice 1',
...         ),
...     consort.PerformedTimespan(
...         layer=2,
...         start_offset=10,
...         stop_offset=(25, 2),
...         music_specifier=unlabeled_music_specifier,
...         voice_name='Voice 2',
...         ),
...     consort.PerformedTimespan(
...         layer=1,
...         start_offset=11,
...         stop_offset=14,
...         music_specifier=labeled_music_specifier,
...         voice_name='Voice 1',
...         ),
...     consort.PerformedTimespan(
...         layer=1,
...         start_offset=14,
...         stop_offset=16,
...         music_specifier=labeled_music_specifier,
...         voice_name='Voice 1',
...         ),
...     consort.PerformedTimespan(
...         layer=1,
...         start_offset=15,
...         stop_offset=16,
...         music_specifier=labeled_music_specifier,
...         voice_name='Voice 2',
...         ),
...     ])
>>> show(timespan_inventory, key='voice_name')
\end{lstlisting}
\noindent\includegraphics[max width=\textwidth,]{assets/lilypond-8ff4b7fe83d5db8fb0d01627be42da4b.pdf}
\end{singlespacing}
%%% ABJADBOOK END %%%

\begin{comment}
<abjad>
dependent_timespan_maker = consort.DependentTimespanMaker(
    include_inner_starts=True,
    voice_names=('Voice 1', 'Voice 2'),
    )
result = dependent_timespan_maker(
    layer=3,
    music_specifiers={'Voice 3': None},
    timespan_inventory=timespan_inventory[:],
    )
show(result, key='voice_name')
</abjad>
\end{comment}

%%% ABJADBOOK START %%%
\begin{singlespacing}
\vspace{-0.5\baselineskip}
\begin{lstlisting}
>>> dependent_timespan_maker = consort.DependentTimespanMaker(
...     include_inner_starts=True,
...     voice_names=('Voice 1', 'Voice 2'),
...     )
>>> result = dependent_timespan_maker(
...     layer=3,
...     music_specifiers={'Voice 3': None},
...     timespan_inventory=timespan_inventory[:],
...     )
>>> show(result, key='voice_name')
\end{lstlisting}
\noindent\includegraphics[max width=\textwidth,]{assets/lilypond-cacf84e69a9aeb754e2908597bc776a2.pdf}
\end{singlespacing}
%%% ABJADBOOK END %%%

\begin{comment}
<abjad>
dependent_timespan_maker = new(
    dependent_timespan_maker,
    labels=['labeled'],
    )
result = dependent_timespan_maker(
    layer=3,
    music_specifiers={'Voice 3': None},
    timespan_inventory=timespan_inventory[:],
    )
show(result, key='voice_name')
</abjad>
\end{comment}

%%% ABJADBOOK START %%%
\begin{singlespacing}
\vspace{-0.5\baselineskip}
\begin{lstlisting}
>>> dependent_timespan_maker = new(
...     dependent_timespan_maker,
...     labels=['labeled'],
...     )
>>> result = dependent_timespan_maker(
...     layer=3,
...     music_specifiers={'Voice 3': None},
...     timespan_inventory=timespan_inventory[:],
...     )
>>> show(result, key='voice_name')
\end{lstlisting}
\noindent\includegraphics[max width=\textwidth,]{assets/lilypond-9382d23b443fa753fb33e78d0e7a805f.pdf}
\end{singlespacing}
%%% ABJADBOOK END %%%

\begin{comment}
\begin{markdown}
-   Attachment-handlers:
    -   Attaches things to the score.
    -   Aggregates AttachmentExpression instances together.
        -   A bundle of a selector and an iterable of attachments or
            expressions.
        -   Reprise discussion of selectors.
        -   Discuss expressions: DynamicExpression, etc.
    -   Processes the score by voice, then by *phrase*.
        -   Unlike the other handlers, attachment handlers require the entire
            phrase to operate on, and selectors should be designed with that in
            mind.
    -   Demonstrate a gallery of selectors.
        -   by duration
        -   by leaves
        -   by length
        -   by logical tie
        -   by counts (with negative counts too)
    -   Expressive attachments
        -   Idiomatic indicators
        -   DynamicExpression
        -   BowContactSpanner
        -   StringContactSpanner
-   Other music specifier properties of note?
    -   *labels*
        -   This works in tandem with DependentTimespanMaker.
        -   Example? Piano music with two hands and pedaling. The
            hand-performed music might involve key presses, or it might involve
            percussive techniques. All of the techniques that require pedaling
            should be labeled 'pedaled'. A DependentTimespanMaker for the pedal
            context can then be configured to look at timespans for both the LH
            and RH of the piano, but only those timespans configured with a
            music specifier labeled 'pedaled'.
        -   Demonstrate.
    -   *minimum phrase duration*
        -   (should this be hoisted into PerformedTimespan?
    -   *seed*
    -   *is_sentinel* (maybe remove from consort altogether)
\end{markdown}
\end{comment}

\section{Composite music specifiers}

\section{Timespan population}

\begin{markdown}
-   Multiplexing & demultiplexing
-   Resolving cascading overlap
\end{markdown}

\subsection{Timespan resolution examples}

\begin{comment}
<abjad>
layer_1_timespan_maker = consort.FloodedTimespanMaker()
layer_1_target_timespan = timespantools.Timespan(0, (19, 4))
layer_1_music_specifiers = collections.OrderedDict([
    ('Voice 2', None),
    ('Voice 3', None),
    ])
layer_1 = layer_1_timespan_maker(
    layer=1,
    music_specifiers=layer_1_music_specifiers,
    target_timespan=layer_1_target_timespan,
    )
show(layer_1, key='voice_name', range_=(0, (21, 4)))
</abjad>
\end{comment}

%%% ABJADBOOK START %%%
\begin{singlespacing}
\vspace{-0.5\baselineskip}
\begin{lstlisting}
>>> layer_1_timespan_maker = consort.FloodedTimespanMaker()
>>> layer_1_target_timespan = timespantools.Timespan(0, (19, 4))
>>> layer_1_music_specifiers = collections.OrderedDict([
...     ('Voice 2', None),
...     ('Voice 3', None),
...     ])
>>> layer_1 = layer_1_timespan_maker(
...     layer=1,
...     music_specifiers=layer_1_music_specifiers,
...     target_timespan=layer_1_target_timespan,
...     )
>>> show(layer_1, key='voice_name', range_=(0, (21, 4)))
\end{lstlisting}
\noindent\includegraphics[max width=\textwidth,]{assets/lilypond-f86c3ef83d7c5ffbef9e47999bb478a9.pdf}
\end{singlespacing}
%%% ABJADBOOK END %%%

\begin{comment}
<abjad>
layer_2_timespan_maker = consort.TaleaTimespanMaker(
    initial_silence_talea=rhythmmakertools.Talea(
        counts=(0, 1, 3),
        denominator=8,
        ),
    playing_groupings=(1, 2),
    playing_talea=rhythmmakertools.Talea(
        counts=(1, 2, 3, 4),
        denominator=4,
        ),
    silence_talea=rhythmmakertools.Talea(
        counts=(5, 3, 1),
        denominator=8,
        ),
    )
layer_2_target_timespan = timespantools.Timespan((3, 4), (19, 4))
layer_2_music_specifiers = collections.OrderedDict([
    ('Voice 1', None),
    ('Voice 2', None),
    ('Voice 3', None),
    ('Voice 4', None),
    ])
layer_2 = layer_2_timespan_maker(
    layer=2,
    music_specifiers=layer_2_music_specifiers,
    target_timespan=layer_2_target_timespan,
    )
show(layer_2, key='voice_name', range_=(0, (21, 4)))
</abjad>
\end{comment}

%%% ABJADBOOK START %%%
\begin{singlespacing}
\vspace{-0.5\baselineskip}
\begin{lstlisting}
>>> layer_2_timespan_maker = consort.TaleaTimespanMaker(
...     initial_silence_talea=rhythmmakertools.Talea(
...         counts=(0, 1, 3),
...         denominator=8,
...         ),
...     playing_groupings=(1, 2),
...     playing_talea=rhythmmakertools.Talea(
...         counts=(1, 2, 3, 4),
...         denominator=4,
...         ),
...     silence_talea=rhythmmakertools.Talea(
...         counts=(5, 3, 1),
...         denominator=8,
...         ),
...     )
>>> layer_2_target_timespan = timespantools.Timespan((3, 4), (19, 4))
>>> layer_2_music_specifiers = collections.OrderedDict([
...     ('Voice 1', None),
...     ('Voice 2', None),
...     ('Voice 3', None),
...     ('Voice 4', None),
...     ])
>>> layer_2 = layer_2_timespan_maker(
...     layer=2,
...     music_specifiers=layer_2_music_specifiers,
...     target_timespan=layer_2_target_timespan,
...     )
>>> show(layer_2, key='voice_name', range_=(0, (21, 4)))
\end{lstlisting}
\noindent\includegraphics[max width=\textwidth,]{assets/lilypond-e49b088ae657fd66299d7b78da251b1a.pdf}
\end{singlespacing}
%%% ABJADBOOK END %%%

\begin{comment}
<abjad>
layer_3_timespan_maker = consort.TaleaTimespanMaker(
    initial_silence_talea=rhythmmakertools.Talea(
        counts=(0, 0, 0, 1),
        denominator=8,
        ),
    padding=Duration(1, 4),
    playing_talea=rhythmmakertools.Talea(
        counts=(2, 3, 4),
        denominator=8,
        ),
    silence_talea=rhythmmakertools.Talea(
        counts=(6,),
        denominator=4,
        ),
    synchronize_step=True,
    )
layer_3_target_timespan = timespantools.Timespan((6, 4), (21, 4))
layer_3_music_specifiers = collections.OrderedDict([
    ('Voice 1', None),
    ('Voice 3', None),
    ('Voice 4', None),
    ])
layer_3 = layer_3_timespan_maker(
    layer=3,
    music_specifiers=layer_3_music_specifiers,
    target_timespan=layer_3_target_timespan,
    )
show(layer_3, key='voice_name', range_=(0, (21, 4)))
</abjad>
\end{comment}

%%% ABJADBOOK START %%%
\begin{singlespacing}
\vspace{-0.5\baselineskip}
\begin{lstlisting}
>>> layer_3_timespan_maker = consort.TaleaTimespanMaker(
...     initial_silence_talea=rhythmmakertools.Talea(
...         counts=(0, 0, 0, 1),
...         denominator=8,
...         ),
...     padding=Duration(1, 4),
...     playing_talea=rhythmmakertools.Talea(
...         counts=(2, 3, 4),
...         denominator=8,
...         ),
...     silence_talea=rhythmmakertools.Talea(
...         counts=(6,),
...         denominator=4,
...         ),
...     synchronize_step=True,
...     )
>>> layer_3_target_timespan = timespantools.Timespan((6, 4), (21, 4))
>>> layer_3_music_specifiers = collections.OrderedDict([
...     ('Voice 1', None),
...     ('Voice 3', None),
...     ('Voice 4', None),
...     ])
>>> layer_3 = layer_3_timespan_maker(
...     layer=3,
...     music_specifiers=layer_3_music_specifiers,
...     target_timespan=layer_3_target_timespan,
...     )
>>> show(layer_3, key='voice_name', range_=(0, (21, 4)))
\end{lstlisting}
\noindent\includegraphics[max width=\textwidth,]{assets/lilypond-941b5746ac1f0adcadef1f907d8ffb78.pdf}
\end{singlespacing}
%%% ABJADBOOK END %%%

\begin{comment}
<abjad>
timespan_makers = (
    layer_1_timespan_maker,
    layer_2_timespan_maker,
    layer_3_timespan_maker,
    )
music_specifiers = (
    layer_1_music_specifiers,
    layer_2_music_specifiers,
    layer_3_music_specifiers,
    )
target_timespans = (
    layer_1_target_timespan,
    layer_2_target_timespan,
    layer_3_target_timespan,
    )
triples = zip(timespan_makers, music_specifiers, target_timespans)
timespan_inventory = timespantools.TimespanInventory()
for layer, triple in enumerate(triples, 1):
    timespan_inventory = triple[0](
        layer=layer,
        music_specifiers=triple[1],
        target_timespan=triple[2],
        timespan_inventory=timespan_inventory,
        )

show(timespan_inventory, key='voice_name', range_=(0, (21, 4)))
</abjad>
\end{comment}

%%% ABJADBOOK START %%%
\begin{singlespacing}
\vspace{-0.5\baselineskip}
\begin{lstlisting}
>>> timespan_makers = (
...     layer_1_timespan_maker,
...     layer_2_timespan_maker,
...     layer_3_timespan_maker,
...     )
>>> music_specifiers = (
...     layer_1_music_specifiers,
...     layer_2_music_specifiers,
...     layer_3_music_specifiers,
...     )
>>> target_timespans = (
...     layer_1_target_timespan,
...     layer_2_target_timespan,
...     layer_3_target_timespan,
...     )
>>> triples = zip(timespan_makers, music_specifiers, target_timespans)
>>> timespan_inventory = timespantools.TimespanInventory()
>>> for layer, triple in enumerate(triples, 1):
...     timespan_inventory = triple[0](
...         layer=layer,
...         music_specifiers=triple[1],
...         target_timespan=triple[2],
...         timespan_inventory=timespan_inventory,
...         )
...
>>> show(timespan_inventory, key='voice_name', range_=(0, (21, 4)))
\end{lstlisting}
\noindent\includegraphics[max width=\textwidth,]{assets/lilypond-cd13e1c03522127ffbfba2623f59a7fb.pdf}
\end{singlespacing}
%%% ABJADBOOK END %%%

\begin{comment}
<abjad>
show(timespan_inventory, range_=(0, (21, 4)))
</abjad>
\end{comment}

%%% ABJADBOOK START %%%
\begin{singlespacing}
\vspace{-0.5\baselineskip}
\begin{lstlisting}
>>> show(timespan_inventory, range_=(0, (21, 4)))
\end{lstlisting}
\noindent\includegraphics[max width=\textwidth,]{assets/lilypond-39bc7fc954dbcc692b2517c8195b821e.pdf}
\end{singlespacing}
%%% ABJADBOOK END %%%

\begin{comment}
<abjad>
demultiplexed_timespans = consort.TimeManager.demultiplex_timespans(
    timespan_inventory,
    )
show(demultiplexed_timespans, range_=(0, (21, 4)))
</abjad>
\end{comment}

%%% ABJADBOOK START %%%
\begin{singlespacing}
\vspace{-0.5\baselineskip}
\begin{lstlisting}
>>> demultiplexed_timespans = consort.TimeManager.demultiplex_timespans(
...     timespan_inventory,
...     )
>>> show(demultiplexed_timespans, range_=(0, (21, 4)))
\end{lstlisting}
\noindent\includegraphics[max width=\textwidth,]{assets/lilypond-8d7921006d6bce021d9a284d12ed2e90.pdf}
\end{singlespacing}
%%% ABJADBOOK END %%%

\section{Timespan Inscription}

\begin{markdown}
-   Divisions
-   Rhythm creation
-   Rest consolidation
-   Rhythmic post-processing
    -   attaching a GeneralizedBeam
    -   attaching the scoped music specifier
-   Meter rewriting
    Cleaning up logical ties
-   Attack-point collection
\end{markdown}

\subsection{Rest consolidation examples}

\begin{comment}
<abjad>
parseable = r'''
\new Voice {
    {
        {
            \time 4/4
            r4
            c4
        }
        {
            \times 2/3 {
                c4
                r8
            }
            r4
        }
        {
            \time 4/4
            r4
            c4.
            r8
        }
        {
            r4
            \break
        }
        {
            \time 3/4
            r4
        }
        {
            c8
            c4.
        }
        \times 2/3 {
            \time 2/4
            r4
            c4
            c4
        }
        {
            \time 4/4
            c4.
            r8
        }
        {
            r8
            c4
            r8
        }
    }
}
'''
unconsolidated_staff = Staff(parseable, context_name='RhythmicStaff')
consort.annotate(unconsolidated_staff)
show(unconsolidated_staff)
</abjad>
\end{comment}

%%% ABJADBOOK START %%%
\begin{singlespacing}
\vspace{-0.5\baselineskip}
\begin{lstlisting}
>>> parseable = r'''
... \new Voice {
...     {
...         {
...             \time 4/4
...             r4
...             c4
...         }
...         {
...             \times 2/3 {
...                 c4
...                 r8
...             }
...             r4
...         }
...         {
...             \time 4/4
...             r4
...             c4.
...             r8
...         }
...         {
...             r4
...             \break
...         }
...         {
...             \time 3/4
...             r4
...         }
...         {
...             c8
...             c4.
...         }
...         \times 2/3 {
...             \time 2/4
...             r4
...             c4
...             c4
...         }
...         {
...             \time 4/4
...             c4.
...             r8
...         }
...         {
...             r8
...             c4
...             r8
...         }
...     }
... }
... '''
>>> unconsolidated_staff = Staff(parseable, context_name='RhythmicStaff')
>>> consort.annotate(unconsolidated_staff)
>>> show(unconsolidated_staff)
\end{lstlisting}
\noindent\includegraphics[max width=\textwidth,]{assets/lilypond-8a213398dc0a03a0c9fcf4af26637767.pdf}
\end{singlespacing}
%%% ABJADBOOK END %%%

\begin{comment}
<abjad>
consolidated_staff = Staff(parseable, context_name='RhythmicStaff')
for voice in consolidated_staff:
    for phrase in voice:
        phrase = consort.TimeManager.consolidate_rests(phrase)

consort.annotate(consolidated_staff)
staff_group = StaffGroup([unconsolidated_staff, consolidated_staff])
show(staff_group)
</abjad>
\end{comment}

%%% ABJADBOOK START %%%
\begin{singlespacing}
\vspace{-0.5\baselineskip}
\begin{lstlisting}
>>> consolidated_staff = Staff(parseable, context_name='RhythmicStaff')
>>> for voice in consolidated_staff:
...     for phrase in voice:
...         phrase = consort.TimeManager.consolidate_rests(phrase)
...
>>> consort.annotate(consolidated_staff)
>>> staff_group = StaffGroup([unconsolidated_staff, consolidated_staff])
>>> show(staff_group)
\end{lstlisting}
\noindent\includegraphics[max width=\textwidth,]{assets/lilypond-1ef1a4031465a364425aef4e059efda0.pdf}
\end{singlespacing}
%%% ABJADBOOK END %%%


\begin{markdown}
# Post-rhythm-creation processing
-   Score traversal revisited
-   Grace handling
-   Pitch handling
    -   SeedSession
    -   Pitch operations
    -   Logical tie expressions
    -   Pitch application rate
    -   Pitch specifier
    -   Grace expressions
-   Attachment handling
\end{markdown}

\subsection{Pitch handler examples}

\begin{comment}
<abjad>[stylesheet=../consort.ily]
segment_maker = consort.SegmentMaker(
    desired_duration_in_seconds=9,
    omit_stylesheets=True,
    permitted_time_signatures=[(3, 4)],
    score_template=templatetools.GroupedStavesScoreTemplate(
        staff_count=2,
        ),
    tempo=indicatortools.Tempo((1, 4), 60),
    )
music_specifier = consort.MusicSpecifier(
    rhythm_maker=rhythmmakertools.TaleaRhythmMaker(
        talea=rhythmmakertools.Talea([1], 16),
        ),
    )
timespan_maker = consort.TaleaTimespanMaker(
    initial_silence_talea=rhythmmakertools.Talea([0, 1], 4),
    playing_talea=rhythmmakertools.Talea([1], 8),
    playing_groupings=[3],
    silence_talea=rhythmmakertools.Talea([1], 8),
    )
segment_maker.add_setting(
    timespan_maker=timespan_maker,
    v1=music_specifier,
    v2=music_specifier,
    )
show(segment_maker, verbose=False)
</abjad>
\end{comment}

%%% ABJADBOOK START %%%
\begin{singlespacing}
\vspace{-0.5\baselineskip}
\begin{lstlisting}
>>> segment_maker = consort.SegmentMaker(
...     desired_duration_in_seconds=9,
...     omit_stylesheets=True,
...     permitted_time_signatures=[(3, 4)],
...     score_template=templatetools.GroupedStavesScoreTemplate(
...         staff_count=2,
...         ),
...     tempo=indicatortools.Tempo((1, 4), 60),
...     )
>>> music_specifier = consort.MusicSpecifier(
...     rhythm_maker=rhythmmakertools.TaleaRhythmMaker(
...         talea=rhythmmakertools.Talea([1], 16),
...         ),
...     )
>>> timespan_maker = consort.TaleaTimespanMaker(
...     initial_silence_talea=rhythmmakertools.Talea([0, 1], 4),
...     playing_talea=rhythmmakertools.Talea([1], 8),
...     playing_groupings=[3],
...     silence_talea=rhythmmakertools.Talea([1], 8),
...     )
>>> segment_maker.add_setting(
...     timespan_maker=timespan_maker,
...     v1=music_specifier,
...     v2=music_specifier,
...     )
>>> show(segment_maker, verbose=False)
\end{lstlisting}
\noindent\includegraphics[max width=\textwidth,]{assets/lilypond-394f358f60132ecf989d89a8a9e62012.pdf}
\end{singlespacing}
%%% ABJADBOOK END %%%

\begin{comment}
<abjad>[stylesheet=../consort.ily]
music_specifier = new(
    music_specifier,
    pitch_handler=consort.AbsolutePitchHandler(
        pitch_specifier="c' d' e' f' g' a' b' c''",
        ),
    )
music_setting = consort.MusicSetting(
    timespan_maker=timespan_maker,
    v1=music_specifier,
    v2=music_specifier,
    )
segment_maker = new(segment_maker, settings=[music_setting])
show(segment_maker, verbose=False)
</abjad>
\end{comment}

%%% ABJADBOOK START %%%
\begin{singlespacing}
\vspace{-0.5\baselineskip}
\begin{lstlisting}
>>> music_specifier = new(
...     music_specifier,
...     pitch_handler=consort.AbsolutePitchHandler(
...         pitch_specifier="c' d' e' f' g' a' b' c''",
...         ),
...     )
>>> music_setting = consort.MusicSetting(
...     timespan_maker=timespan_maker,
...     v1=music_specifier,
...     v2=music_specifier,
...     )
>>> segment_maker = new(segment_maker, settings=[music_setting])
>>> show(segment_maker, verbose=False)
\end{lstlisting}
\noindent\includegraphics[max width=\textwidth,]{assets/lilypond-104b2b0b061ac0a12d87191eadfae379.pdf}
\end{singlespacing}
%%% ABJADBOOK END %%%

\begin{comment}
<abjad>[stylesheet=../consort.ily]
music_specifier = new(
    music_specifier,
    rhythm_maker__talea__counts=[1, 2, 3, 4],
    )
music_setting = consort.MusicSetting(
    timespan_maker=timespan_maker,
    v1=music_specifier,
    v2=music_specifier,
    )
segment_maker = new(segment_maker, settings=[music_setting])
show(segment_maker, verbose=False)
</abjad>
\end{comment}

%%% ABJADBOOK START %%%
\begin{singlespacing}
\vspace{-0.5\baselineskip}
\begin{lstlisting}
>>> music_specifier = new(
...     music_specifier,
...     rhythm_maker__talea__counts=[1, 2, 3, 4],
...     )
>>> music_setting = consort.MusicSetting(
...     timespan_maker=timespan_maker,
...     v1=music_specifier,
...     v2=music_specifier,
...     )
>>> segment_maker = new(segment_maker, settings=[music_setting])
>>> show(segment_maker, verbose=False)
\end{lstlisting}
\noindent\includegraphics[max width=\textwidth,]{assets/lilypond-1548051ace14e831b1685f04ffc9bf04.pdf}
\end{singlespacing}
%%% ABJADBOOK END %%%

\begin{comment}
<abjad>[stylesheet=../consort.ily]
other_music_specifier = consort.MusicSpecifier(
    pitch_handler=consort.AbsolutePitchHandler(pitch_specifier='g fs e f'),
    rhythm_maker=rhythmmakertools.EvenDivisionRhythmMaker(
        denominators=[8],
        extra_counts_per_division=(1,),
        ),
    )
other_music_setting = consort.MusicSetting(
    timespan_maker=consort.TaleaTimespanMaker(
        initial_silence_talea=rhythmmakertools.Talea([1], 2),
        silence_talea=rhythmmakertools.Talea([1], 2),
        ),
    v1=other_music_specifier,
    v2=other_music_specifier,
    )
segment_maker = new(
    segment_maker,
    settings=[music_setting, other_music_setting],
    )
show(segment_maker, verbose=False)
</abjad>
\end{comment}

%%% ABJADBOOK START %%%
\begin{singlespacing}
\vspace{-0.5\baselineskip}
\begin{lstlisting}
>>> other_music_specifier = consort.MusicSpecifier(
...     pitch_handler=consort.AbsolutePitchHandler(pitch_specifier='g fs e f'),
...     rhythm_maker=rhythmmakertools.EvenDivisionRhythmMaker(
...         denominators=[8],
...         extra_counts_per_division=(1,),
...         ),
...     )
>>> other_music_setting = consort.MusicSetting(
...     timespan_maker=consort.TaleaTimespanMaker(
...         initial_silence_talea=rhythmmakertools.Talea([1], 2),
...         silence_talea=rhythmmakertools.Talea([1], 2),
...         ),
...     v1=other_music_specifier,
...     v2=other_music_specifier,
...     )
>>> segment_maker = new(
...     segment_maker,
...     settings=[music_setting, other_music_setting],
...     )
>>> show(segment_maker, verbose=False)
\end{lstlisting}
\noindent\includegraphics[max width=\textwidth,]{assets/lilypond-e6f73fdc68fe0c974a8ca8492d683bdb.pdf}
\end{singlespacing}
%%% ABJADBOOK END %%%

\begin{comment}
<abjad>[stylesheet=../consort.ily]
music_specifier = new(
    music_specifier,
    pitch_handler__pitch_specifier=consort.PitchSpecifier(
        pitch_segments=(
            "c' e' g'",
            "fs' g' a'",
            "b d'",
            ),
        ratio=(1, 2, 3),
        ),
    )
music_setting = consort.MusicSetting(
    timespan_maker=timespan_maker,
    v1=music_specifier,
    v2=music_specifier,
    )
segment_maker = new(segment_maker, settings=[music_setting])
show(segment_maker, verbose=False)
</abjad>
\end{comment}

%%% ABJADBOOK START %%%
\begin{singlespacing}
\vspace{-0.5\baselineskip}
\begin{lstlisting}
>>> music_specifier = new(
...     music_specifier,
...     pitch_handler__pitch_specifier=consort.PitchSpecifier(
...         pitch_segments=(
...             "c' e' g'",
...             "fs' g' a'",
...             "b d'",
...             ),
...         ratio=(1, 2, 3),
...         ),
...     )
>>> music_setting = consort.MusicSetting(
...     timespan_maker=timespan_maker,
...     v1=music_specifier,
...     v2=music_specifier,
...     )
>>> segment_maker = new(segment_maker, settings=[music_setting])
>>> show(segment_maker, verbose=False)
\end{lstlisting}
\noindent\includegraphics[max width=\textwidth,]{assets/lilypond-f863612768a505f3f1a27901d8908c01.pdf}
\end{singlespacing}
%%% ABJADBOOK END %%%

\begin{comment}
<abjad>[stylesheet=../consort.ily]
music_specifier = new(
    music_specifier,
    pitch_handler__pitch_operation_specifier=consort.PitchOperationSpecifier(
        pitch_operations=(
            pitchtools.PitchOperation((
                pitchtools.Inversion(),
                )),
            None,
            pitchtools.PitchOperation((
                pitchtools.Rotation(-1),
                pitchtools.Transposition(-1),
                ))
            ),
        ratio=(1, 2, 1),
        ),
    )
music_setting = consort.MusicSetting(
    timespan_maker=timespan_maker,
    v1=music_specifier,
    v2=music_specifier,
    )
segment_maker = new(segment_maker, settings=[music_setting])
show(segment_maker, verbose=False)
</abjad>
\end{comment}

%%% ABJADBOOK START %%%
\begin{singlespacing}
\vspace{-0.5\baselineskip}
\begin{lstlisting}
>>> music_specifier = new(
...     music_specifier,
...     pitch_handler__pitch_operation_specifier=consort.PitchOperationSpecifier(
...         pitch_operations=(
...             pitchtools.PitchOperation((
...                 pitchtools.Inversion(),
...                 )),
...             None,
...             pitchtools.PitchOperation((
...                 pitchtools.Rotation(-1),
...                 pitchtools.Transposition(-1),
...                 ))
...             ),
...         ratio=(1, 2, 1),
...         ),
...     )
>>> music_setting = consort.MusicSetting(
...     timespan_maker=timespan_maker,
...     v1=music_specifier,
...     v2=music_specifier,
...     )
>>> segment_maker = new(segment_maker, settings=[music_setting])
>>> show(segment_maker, verbose=False)
\end{lstlisting}
\noindent\includegraphics[max width=\textwidth,]{assets/lilypond-8353cb721ae26e66270945a62cbb8a67.pdf}
\end{singlespacing}
%%% ABJADBOOK END %%%

\begin{comment}
<abjad>[stylesheet=../consort.ily]
music_specifier = new(
    music_specifier,
    pitch_handler__forbid_repetitions=True,
    )
music_setting = consort.MusicSetting(
    timespan_maker=timespan_maker,
    v1=music_specifier,
    v2=music_specifier,
    )
segment_maker = new(segment_maker, settings=[music_setting])
show(segment_maker, verbose=False)
</abjad>
\end{comment}

%%% ABJADBOOK START %%%
\begin{singlespacing}
\vspace{-0.5\baselineskip}
\begin{lstlisting}
>>> music_specifier = new(
...     music_specifier,
...     pitch_handler__forbid_repetitions=True,
...     )
>>> music_setting = consort.MusicSetting(
...     timespan_maker=timespan_maker,
...     v1=music_specifier,
...     v2=music_specifier,
...     )
>>> segment_maker = new(segment_maker, settings=[music_setting])
>>> show(segment_maker, verbose=False)
\end{lstlisting}
\noindent\includegraphics[max width=\textwidth,]{assets/lilypond-e8337085639c633444a1e65214aa5493.pdf}
\end{singlespacing}
%%% ABJADBOOK END %%%

\begin{comment}
<abjad>[stylesheet=../consort.ily]
music_specifier = new(
    music_specifier,
    pitch_handler__logical_tie_expressions=(
        consort.ChordExpression(chord_expr=(-2, 0, 2)),
        consort.ChordExpression(chord_expr=(-7, 0, 7)),
        None,
        ),
    )
music_setting = consort.MusicSetting(
    timespan_maker=timespan_maker,
    v1=music_specifier,
    v2=music_specifier,
    )
segment_maker = new(segment_maker, settings=[music_setting])
show(segment_maker, verbose=False)
</abjad>
\end{comment}

%%% ABJADBOOK START %%%
\begin{singlespacing}
\vspace{-0.5\baselineskip}
\begin{lstlisting}
>>> music_specifier = new(
...     music_specifier,
...     pitch_handler__logical_tie_expressions=(
...         consort.ChordExpression(chord_expr=(-2, 0, 2)),
...         consort.ChordExpression(chord_expr=(-7, 0, 7)),
...         None,
...         ),
...     )
>>> music_setting = consort.MusicSetting(
...     timespan_maker=timespan_maker,
...     v1=music_specifier,
...     v2=music_specifier,
...     )
>>> segment_maker = new(segment_maker, settings=[music_setting])
>>> show(segment_maker, verbose=False)
\end{lstlisting}
\noindent\includegraphics[max width=\textwidth,]{assets/lilypond-1faec952a7b45f54cee2ed62c44d50bc.pdf}
\end{singlespacing}
%%% ABJADBOOK END %%%

\begin{comment}
<abjad>[stylesheet=../consort.ily]
segment_maker = new(
    segment_maker,
    settings=[music_setting, other_music_setting],
    )
show(segment_maker, verbose=False)
</abjad>
\end{comment}

%%% ABJADBOOK START %%%
\begin{singlespacing}
\vspace{-0.5\baselineskip}
\begin{lstlisting}
>>> segment_maker = new(
...     segment_maker,
...     settings=[music_setting, other_music_setting],
...     )
>>> show(segment_maker, verbose=False)
\end{lstlisting}
\noindent\includegraphics[max width=\textwidth,]{assets/lilypond-9b48c7c7d8ec2ca4e86b9da611fdef17.pdf}
\end{singlespacing}
%%% ABJADBOOK END %%%

\begin{comment}
<abjad>[stylesheet=../consort.ily]
music_specifier = consort.MusicSpecifier(
    pitch_handler=consort.AbsolutePitchHandler(
        pitch_application_rate='division',
        pitch_specifier="c' d' e' f' g' a' b' c''",
        ),
    rhythm_maker=rhythmmakertools.EvenDivisionRhythmMaker(denominators=[16]),
    )
music_setting = consort.MusicSetting(
    timespan_maker=timespan_maker,
    v1=music_specifier,
    v2=music_specifier,
    )
segment_maker = new(segment_maker, settings=[music_setting])
lilypond_file = segment_maker(verbose=False)
consort.annotate(lilypond_file.score)
show(lilypond_file)
</abjad>
\end{comment}

%%% ABJADBOOK START %%%
\begin{singlespacing}
\vspace{-0.5\baselineskip}
\begin{lstlisting}
>>> music_specifier = consort.MusicSpecifier(
...     pitch_handler=consort.AbsolutePitchHandler(
...         pitch_application_rate='division',
...         pitch_specifier="c' d' e' f' g' a' b' c''",
...         ),
...     rhythm_maker=rhythmmakertools.EvenDivisionRhythmMaker(denominators=[16]),
...     )
>>> music_setting = consort.MusicSetting(
...     timespan_maker=timespan_maker,
...     v1=music_specifier,
...     v2=music_specifier,
...     )
>>> segment_maker = new(segment_maker, settings=[music_setting])
>>> lilypond_file = segment_maker(verbose=False)
>>> consort.annotate(lilypond_file.score)
>>> show(lilypond_file)
\end{lstlisting}
\noindent\includegraphics[max width=\textwidth,]{assets/lilypond-76acdc0bda9feecdd70997204aa7c9c8.pdf}
\end{singlespacing}
%%% ABJADBOOK END %%%

\begin{comment}
<abjad>[stylesheet=../consort.ily]
music_specifier = new(
    music_specifier,
    pitch_handler__pitch_application_rate='phrase',
    )
music_setting = consort.MusicSetting(
    timespan_maker=timespan_maker,
    v1=music_specifier,
    v2=music_specifier,
    )
segment_maker = new(segment_maker, settings=[music_setting])
lilypond_file = segment_maker(verbose=False)
consort.annotate(lilypond_file.score)
show(lilypond_file)
</abjad>
\end{comment}

%%% ABJADBOOK START %%%
\begin{singlespacing}
\vspace{-0.5\baselineskip}
\begin{lstlisting}
>>> music_specifier = new(
...     music_specifier,
...     pitch_handler__pitch_application_rate='phrase',
...     )
>>> music_setting = consort.MusicSetting(
...     timespan_maker=timespan_maker,
...     v1=music_specifier,
...     v2=music_specifier,
...     )
>>> segment_maker = new(segment_maker, settings=[music_setting])
>>> lilypond_file = segment_maker(verbose=False)
>>> consort.annotate(lilypond_file.score)
>>> show(lilypond_file)
\end{lstlisting}
\noindent\includegraphics[max width=\textwidth,]{assets/lilypond-89244abdb768aa23fa5a7425fab9e539.pdf}
\end{singlespacing}
%%% ABJADBOOK END %%%

\begin{comment}
<abjad>[stylesheet=../consort.ily]
music_specifier = new(
    music_specifier,
    pitch_handler__deviations=(0, 0, '-m2', '+m2'),
    )
music_setting = consort.MusicSetting(
    timespan_maker=timespan_maker,
    v1=music_specifier,
    v2=music_specifier,
    )
segment_maker = new(segment_maker, settings=[music_setting])
lilypond_file = segment_maker(verbose=False)
consort.annotate(lilypond_file.score)
show(lilypond_file)
</abjad>
\end{comment}

%%% ABJADBOOK START %%%
\begin{singlespacing}
\vspace{-0.5\baselineskip}
\begin{lstlisting}
>>> music_specifier = new(
...     music_specifier,
...     pitch_handler__deviations=(0, 0, '-m2', '+m2'),
...     )
>>> music_setting = consort.MusicSetting(
...     timespan_maker=timespan_maker,
...     v1=music_specifier,
...     v2=music_specifier,
...     )
>>> segment_maker = new(segment_maker, settings=[music_setting])
>>> lilypond_file = segment_maker(verbose=False)
>>> consort.annotate(lilypond_file.score)
>>> show(lilypond_file)
\end{lstlisting}
\noindent\includegraphics[max width=\textwidth,]{assets/lilypond-ff0b1c46ddbd29c5a690fe7c182da0aa.pdf}
\end{singlespacing}
%%% ABJADBOOK END %%%

\subsection{Attachment handler examples}

\begin{comment}
<abjad>[stylesheet=../consort.ily]
music_specifier = consort.MusicSpecifier(
    attachment_handler=consort.AttachmentHandler(),
    rhythm_maker=rhythmmakertools.TaleaRhythmMaker(
        extra_counts_per_division=(0, 1),
        talea=rhythmmakertools.Talea([1, 2, 3, 1, 4], 16),
        ),
    )
timespan_maker = consort.TaleaTimespanMaker(
    initial_silence_talea=rhythmmakertools.Talea([0, 1], 4),
    playing_groupings=(1, 2, 2),
    playing_talea=rhythmmakertools.Talea([2, 3], 8),
    silence_talea=rhythmmakertools.Talea([1, 2, 3, 4], 8),
    )
music_setting = consort.MusicSetting(
    timespan_maker=timespan_maker,
    v1=music_specifier,
    v2=music_specifier,
    )
segment_maker = consort.SegmentMaker(
    desired_duration_in_seconds=8,
    discard_final_silence=True,
    omit_stylesheets=True,
    permitted_time_signatures=[(2, 4), (5, 16), (3, 4)],
    score_template=templatetools.GroupedRhythmicStavesScoreTemplate(
        staff_count=2,
        with_clefs=True,
        ),
    settings=[music_setting],
    tempo=indicatortools.Tempo((1, 4), 72),
    )
show(segment_maker, verbose=False)
</abjad>
\end{comment}

%%% ABJADBOOK START %%%
\begin{singlespacing}
\vspace{-0.5\baselineskip}
\begin{lstlisting}
>>> music_specifier = consort.MusicSpecifier(
...     attachment_handler=consort.AttachmentHandler(),
...     rhythm_maker=rhythmmakertools.TaleaRhythmMaker(
...         extra_counts_per_division=(0, 1),
...         talea=rhythmmakertools.Talea([1, 2, 3, 1, 4], 16),
...         ),
...     )
>>> timespan_maker = consort.TaleaTimespanMaker(
...     initial_silence_talea=rhythmmakertools.Talea([0, 1], 4),
...     playing_groupings=(1, 2, 2),
...     playing_talea=rhythmmakertools.Talea([2, 3], 8),
...     silence_talea=rhythmmakertools.Talea([1, 2, 3, 4], 8),
...     )
>>> music_setting = consort.MusicSetting(
...     timespan_maker=timespan_maker,
...     v1=music_specifier,
...     v2=music_specifier,
...     )
>>> segment_maker = consort.SegmentMaker(
...     desired_duration_in_seconds=8,
...     discard_final_silence=True,
...     omit_stylesheets=True,
...     permitted_time_signatures=[(2, 4), (5, 16), (3, 4)],
...     score_template=templatetools.GroupedRhythmicStavesScoreTemplate(
...         staff_count=2,
...         with_clefs=True,
...         ),
...     settings=[music_setting],
...     tempo=indicatortools.Tempo((1, 4), 72),
...     )
>>> show(segment_maker, verbose=False)
\end{lstlisting}
\noindent\includegraphics[max width=\textwidth,]{assets/lilypond-31ad7eaab600ca94fc940e6c050a8651.pdf}
\end{singlespacing}
%%% ABJADBOOK END %%%

\begin{comment}
<abjad>[stylesheet=../consort.ily]
music_specifier = new(
    music_specifier,
    attachment_handler__accents=consort.AttachmentExpression(
        attachments=Articulation('accent'),
        selector=selectortools.Selector().by_leaves()[0],
        ),
    )
music_setting = new(
    music_setting,
    v1=music_specifier,
    v2=music_specifier,
    )
segment_maker = new(segment_maker, settings=[music_setting])
show(segment_maker,verbose=False)
</abjad>
\end{comment}

%%% ABJADBOOK START %%%
\begin{singlespacing}
\vspace{-0.5\baselineskip}
\begin{lstlisting}
>>> music_specifier = new(
...     music_specifier,
...     attachment_handler__accents=consort.AttachmentExpression(
...         attachments=Articulation('accent'),
...         selector=selectortools.Selector().by_leaves()[0],
...         ),
...     )
>>> music_setting = new(
...     music_setting,
...     v1=music_specifier,
...     v2=music_specifier,
...     )
>>> segment_maker = new(segment_maker, settings=[music_setting])
>>> show(segment_maker,verbose=False)
\end{lstlisting}
\noindent\includegraphics[max width=\textwidth,]{assets/lilypond-5845a137b0839d77edd44e425c040cdb.pdf}
\end{singlespacing}
%%% ABJADBOOK END %%%

\begin{comment}
<abjad>[stylesheet=../consort.ily]
music_specifier = new(
    music_specifier,
    attachment_handler__tenuti=consort.AttachmentExpression(
        attachments=Articulation('tenuto'),
        selector=selectortools.Selector()
            .by_leaves()[1:]
            .by_logical_tie(pitched=True)[0],
        ),
    )
music_setting = new(
    music_setting,
    v1=music_specifier,
    v2=music_specifier,
    )
segment_maker = new(segment_maker, settings=[music_setting])
show(segment_maker, verbose=False)
</abjad>
\end{comment}

%%% ABJADBOOK START %%%
\begin{singlespacing}
\vspace{-0.5\baselineskip}
\begin{lstlisting}
>>> music_specifier = new(
...     music_specifier,
...     attachment_handler__tenuti=consort.AttachmentExpression(
...         attachments=Articulation('tenuto'),
...         selector=selectortools.Selector()
...             .by_leaves()[1:]
...             .by_logical_tie(pitched=True)[0],
...         ),
...     )
>>> music_setting = new(
...     music_setting,
...     v1=music_specifier,
...     v2=music_specifier,
...     )
>>> segment_maker = new(segment_maker, settings=[music_setting])
>>> show(segment_maker, verbose=False)
\end{lstlisting}
\noindent\includegraphics[max width=\textwidth,]{assets/lilypond-254b71b1dd161525523b2dfc96ee5b18.pdf}
\end{singlespacing}
%%% ABJADBOOK END %%%

\begin{comment}
<abjad>[stylesheet=../consort.ily]
music_specifier = new(
    music_specifier,
    attachment_handler__slurs=Slur()
    )
music_setting = new(
    music_setting,
    v1=music_specifier,
    v2=music_specifier,
    )
segment_maker = new(segment_maker, settings=[music_setting])
show(segment_maker, verbose=False)
</abjad>
\end{comment}

%%% ABJADBOOK START %%%
\begin{singlespacing}
\vspace{-0.5\baselineskip}
\begin{lstlisting}
>>> music_specifier = new(
...     music_specifier,
...     attachment_handler__slurs=Slur()
...     )
>>> music_setting = new(
...     music_setting,
...     v1=music_specifier,
...     v2=music_specifier,
...     )
>>> segment_maker = new(segment_maker, settings=[music_setting])
>>> show(segment_maker, verbose=False)
\end{lstlisting}
\noindent\includegraphics[max width=\textwidth,]{assets/lilypond-f0e62c688147028c121f9b53bb698a65.pdf}
\end{singlespacing}
%%% ABJADBOOK END %%%

\begin{comment}
<abjad>[stylesheet=../consort.ily]
music_specifier = new(
    music_specifier,
    attachment_handler__dynamics=consort.DynamicExpression(['f', 'p'])
    )
music_setting = new(
    music_setting,
    v1=music_specifier,
    v2=music_specifier,
    )
segment_maker = new(segment_maker, settings=[music_setting])
show(segment_maker, verbose=False)
</abjad>
\end{comment}

%%% ABJADBOOK START %%%
\begin{singlespacing}
\vspace{-0.5\baselineskip}
\begin{lstlisting}
>>> music_specifier = new(
...     music_specifier,
...     attachment_handler__dynamics=consort.DynamicExpression(['f', 'p'])
...     )
>>> music_setting = new(
...     music_setting,
...     v1=music_specifier,
...     v2=music_specifier,
...     )
>>> segment_maker = new(segment_maker, settings=[music_setting])
>>> show(segment_maker, verbose=False)
\end{lstlisting}
\noindent\includegraphics[max width=\textwidth,]{assets/lilypond-5134d656fed6cee151678103491f4f07.pdf}
\end{singlespacing}
%%% ABJADBOOK END %%%

\begin{comment}
<abjad>[stylesheet=../consort.ily]
lilypond_file = segment_maker(verbose=False)
consort.annotate(lilypond_file.score)
show(lilypond_file)
</abjad>
\end{comment}

%%% ABJADBOOK START %%%
\begin{singlespacing}
\vspace{-0.5\baselineskip}
\begin{lstlisting}
>>> lilypond_file = segment_maker(verbose=False)
>>> consort.annotate(lilypond_file.score)
>>> show(lilypond_file)
\end{lstlisting}
\noindent\includegraphics[max width=\textwidth,]{assets/lilypond-6e9d587e5ee37ad6583f5197bed09814.pdf}
\end{singlespacing}
%%% ABJADBOOK END %%%

\begin{markdown}
# Score post-processing
-   Voice copying
\end{markdown}