%%%%%%%%%%%%%%%%%%%%%%%%%%%%%%%%%%%%%%%%%%%%%%%%%%%%%%%%%%%%%%%%%%%%%%%%%%%%%%%
%%%%%%%%%%%%%%%%%%%%%%%%%%%%%%%%%%%%%%%%%%%%%%%%%%%%%%%%%%%%%%%%%%%%%%%%%%%%%%%
\chapter{Conclusion}
\label{chap:conclusion}
%%%%%%%%%%%%%%%%%%%%%%%%%%%%%%%%%%%%%%%%%%%%%%%%%%%%%%%%%%%%%%%%%%%%%%%%%%%%%%%
%%%%%%%%%%%%%%%%%%%%%%%%%%%%%%%%%%%%%%%%%%%%%%%%%%%%%%%%%%%%%%%%%%%%%%%%%%%%%%%

\begin{markdown}

# Future work

In no way do I consider this project finished. Nor do I think a project like
this can ever be complete. There is, in my opinion, no single universal
methodology to composition, nor should there ever be. There is still quite a
lot of work to do to solve an entire array of practical problems, let alone
aesthetic ones.



Having devoted so much effort to large- and small-scale time structures, I need
to turn my attention toward harmony and orchestra as constraints and
coordinating forces.

-   multi-staff writing, piano music pedaling, staff-change-writing
-   more flexible pitch structuring
    -   shared sonorities between parts
    -   fix problems of pitch handler hashing
    -   separate pitch-handler into pitch-class, registration- and chord-
    -   new pitch handlers
    -   modeling of counterpoint rules and dissonance
        - obviously is there is a tremendous amount of research into this
          the question is not of finding something new, but of studying the 
          available research and evaluating various implementations
-   idiomatic writing and awareness
    -   improve string writing based on fingerings and string numbers
    -   provide models of wind multiphonics
    -   test and warn for difficult or impossible wind trills and breathing
    -   improve instrument-specific dynamic handling
-   explicit modeling of variation / transformation between parts in a
    polyphonic texture

There is, I think, always a negotiation between making the "truest" model and
simply getting the job done.

# Parting words

My intention with providing the complete sources to both my scores and working
methods is not that others copy me, although they certainly can if they like. I
would not be offended, but maybe a little disappointed that someone who managed
to put together the tools and knowledge was also so strangely lazy that they
didn't take the time to place some personal stamp on their duplication by
turning the knobs or mixing the potions differently. But that seems incredibly
unlikely to me.

Rather, I want this shared knowledge to be something like a lifeboat to those
who come after me, rather than hiding it away to collect dust.

I imagine myself, a composer ten years younger, searching for answers to many
of the questions I've now made good progress on solving, questions which rarely
even concern making art because there is still too much groundwork to lay. Had
I a clear path then to follow, change, or even wholly react against,
maybe I would have produced more music by now. Or maybe not. I no longer have
the same misgivings as I did when younger about splitting my creative energies
between composition and engineering. If anything, they are now the same
activity.

While the code presented here may become dusty, as most code does, the
concepts and techniques -- as separate from their concrete implementations --
most likely won't. While everything has been presented in Python, this work
could, and should, be implemented in many other programming languages.

\end{markdown}