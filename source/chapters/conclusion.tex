%%%%%%%%%%%%%%%%%%%%%%%%%%%%%%%%%%%%%%%%%%%%%%%%%%%%%%%%%%%%%%%%%%%%%%%%%%%%%%%
%%%%%%%%%%%%%%%%%%%%%%%%%%%%%%%%%%%%%%%%%%%%%%%%%%%%%%%%%%%%%%%%%%%%%%%%%%%%%%%
\chapter{Conclusion}
\label{chap:conclusion}
%%%%%%%%%%%%%%%%%%%%%%%%%%%%%%%%%%%%%%%%%%%%%%%%%%%%%%%%%%%%%%%%%%%%%%%%%%%%%%%
%%%%%%%%%%%%%%%%%%%%%%%%%%%%%%%%%%%%%%%%%%%%%%%%%%%%%%%%%%%%%%%%%%%%%%%%%%%%%%%

\begin{markdown}

# Composing in text

-   Allography
-   Version control
-   If the input is versioned text, not graphic, there is always a clear,
    retrievable history of the thought process behind the work

# Maquettes and unfolding processes

-   Timespans allow for large scale phrasing and density structures.
-   They simply model attack points and durations, making attack and overlap
    density easy to play with.
-   Timespan annotations allow for maquette-like composition techniques, or
    working in a way that conceptually mirrors working with a DAW, even though
    there is no real graphic output except for notation itself.
-   Annotations can act as a description of what sort of material should appear
    in what locations -- both horizontally in time and vertically in the score
    hierarchy.
-   Timespans also allow for overlap, and have affordances for masking.
-   This allows timespans to be created by a factories in layers, and those
    layers resolved down to a single non-overlapping layer via masking.

# Patterns and randomness

-   All of the maker tools described here (and in the next chapter) emphasize
    working with patterns.
-   Sufficiently complex patterns of discrete numbers (and combinations of
    patterns) are perceptually indistinguishable from randomness. It doesn't
    take much.
-   Likewise, randomness -- no matter how it is arrived at, windowed, weighted
    and so forth -- is a kind of maintenance burden. How can you know that
    your compositional machinery is behaving correctly when the results are
    always different by design?
-   Why the extensive use of patterns rather than anything else?
-   Patterns are sufficient
-   They are also testable

# Architecture and extensibility

-   Only a few recipes for timespan and rhythm creation have been outlined
    here. Obviously, many more are possible, with other paradigms. The models
    described here generally rely on patterns. Other work could create
    rhythm-makers which take pitches as input, or which follow breakpoints, or
    mimic some other external structure. Likewise with timespan-makers. More
    work could be done to create stronger intentional interrelations between
    timespans.
-   One of the strengths in the overall design philosophy described throughout
    this chapter is extensibility. Just because I *haven't* yet created a
    derived-from-bird-song rhythm-maker doesn't mean I can't. Such a maker
    could be created and follow the same interface as all other rhythm-makers,
    benefiting from their foundation of shared notational logics.
-   More factories are possible
-   Those outlined here are neither the *only* nor the *best* possible
    factories one could design
-   New factories can make use of the interfaces implemented by these
    factory families (timespan- and rhythm-makers)
-   New factories can also propose entirely new ways of working
    (such as...)
-   Extensibility is afforded because the basic building blocks --
    timespans, timespan inventories, meter and Abjad's model of score --
    are rock solid
-   This way of working also has implications about composing algorithms vs
    composing *with* algorithms
-   The segment-maker (and friends) are designed as black boxes, but also - for
    me, anyways - aren't actually
-   Could someone else even use this system? Maybe, maybe not. I'm more
    concerned with teaching people design principles so that they can construct
    their own compositional models on top of systems such as Abjad.

# Score as evaluable expression, expressivity

-   People always think machines are in control. Hardly.
-   Still, this is about working within formal constraints
-   Avoid fractal act
-   Where to position messiness

\end{markdown}