%%%%%%%%%%%%%%%%%%%%%%%%%%%%%%%%%%%%%%%%%%%%%%%%%%%%%%%%%%%%%%%%%%%%%%%%%%%%%%%
%%%%%%%%%%%%%%%%%%%%%%%%%%%%%%%%%%%%%%%%%%%%%%%%%%%%%%%%%%%%%%%%%%%%%%%%%%%%%%%
\chapter{Conclusion}
\label{chap:conclusion}
%%%%%%%%%%%%%%%%%%%%%%%%%%%%%%%%%%%%%%%%%%%%%%%%%%%%%%%%%%%%%%%%%%%%%%%%%%%%%%%
%%%%%%%%%%%%%%%%%%%%%%%%%%%%%%%%%%%%%%%%%%%%%%%%%%%%%%%%%%%%%%%%%%%%%%%%%%%%%%%

\begin{markdown}

# Future work

In no way do I consider this project finished. Nor do I think a project like
this can ever be complete. There is, in my opinion, no single universal
methodology to composition, nor should there ever be. There is still quite a
lot of work to do to solve an entire array of practical problems, let alone
aesthetic ones.



Having devoted so much effort to large- and small-scale time structures, I need
to turn my attention toward harmony and orchestra as constraints and
coordinating forces.

-   multi-staff writing, piano music pedaling, staff-change-writing
-   more flexible pitch structuring
    -   shared sonorities between parts
    -   fix problems of pitch handler hashing
    -   separate pitch-handler into pitch-class, registration- and chord-
    -   new pitch handlers
    -   modeling of counterpoint rules and dissonance
        - obviously is there is a tremendous amount of research into this
          the question is not of finding something new, but of studying the 
          available research and evaluating various implementations
-   idiomatic writing and awareness
    -   improve string writing based on fingerings and string numbers
    -   provide models of wind multiphonics
    -   test and warn for difficult or impossible wind trills and breathing
    -   improve instrument-specific dynamic handling
-   explicit modeling of variation / transformation between parts in a
    polyphonic texture

There is, I think, always a negotiation between making the "truest" model and
simply getting the job done.



\end{markdown}