%%%%%%%%%%%%%%%%%%%%%%%%%%%%%%%%%%%%%%%%%%%%%%%%%%%%%%%%%%%%%%%%%%%%%%%%%%%%%%%
\chapter{Tools for modeling time, rhythm and meter}
%%%%%%%%%%%%%%%%%%%%%%%%%%%%%%%%%%%%%%%%%%%%%%%%%%%%%%%%%%%%%%%%%%%%%%%%%%%%%%%

\begin{comment}
<abjad>[hide=true]
import consort
</abjad>
\end{comment}

Consort's implementation of a model of composition relies on a number of
different but interrelated models of musical time.

Dichotomies: outside and inside the score hierarchy, with or without regard to
notation, "coarse" versus "fine" or "phrase" versus "event", vertical or
horizontal, metered and unmetered, potentially simultaneous or strictly
contiguous.

Timespans provide a coarse model of musical time, both in and outside of score
hierarchy.

Notated rhythm provides a fine model of musical time, from within score
hierarchy.

Meter coordinates time and rhythm vertically across score hierarchy, and
bridges the coarse and fine stages of rhythmic interpretation.

Meter is generated as a by-product of phrase-level composition. It is not
specified by-hand during composition. This is not out of any desire to valorize
automaticism, but simply because lacking any other compelling reason to
generate a series of meters I felt the best way for myself would be to have
those meters derive from some sort of pre-existent structure in my
compositional process.

A discussion of these time models and their implications will clarify a later
analysis of the implementation of Consort's score interpretation
stage.

%%%%%%%%%%%%%%%%%%%%%%%%%%%%%%%%%%%%%%%%%%%%%%%%%%%%%%%%%%%%%%%%%%%%%%%%%%%%%%%
\section{Timespans}
%%%%%%%%%%%%%%%%%%%%%%%%%%%%%%%%%%%%%%%%%%%%%%%%%%%%%%%%%%%%%%%%%%%%%%%%%%%%%%%

\begin{comment}
<abjad>
timespan = timespantools.Timespan(
    start_offset=Offset(1, 4),
    stop_offset=Offset(3, 2),
    )
</abjad>
\end{comment}

%%% ABJADBOOK START %%%
\begin{singlespacing}
\vspace{-0.5\baselineskip}
\begin{lstlisting}
>>> timespan = timespantools.Timespan(
...     start_offset=Offset(1, 4),
...     stop_offset=Offset(3, 2),
...     )
\end{lstlisting}
\end{singlespacing}
%%% ABJADBOOK END %%%

\begin{comment}
<abjad>
timespan.start_offset
timespan.stop_offset
timespan.duration
timespan.is_well_formed
</abjad>
\end{comment}

%%% ABJADBOOK START %%%
\begin{singlespacing}
\vspace{-0.5\baselineskip}
\begin{lstlisting}
>>> timespan.start_offset
Offset(1, 4)
\end{lstlisting}
\begin{lstlisting}
>>> timespan.stop_offset
Offset(3, 2)
\end{lstlisting}
\begin{lstlisting}
>>> timespan.duration
Duration(5, 4)
\end{lstlisting}
\begin{lstlisting}
>>> timespan.is_well_formed
True
\end{lstlisting}
\end{singlespacing}
%%% ABJADBOOK END %%%

\begin{comment}
<abjad>
malformed_timespan = timespantools.Timespan(0, 0)
malformed_timespan.is_well_formed
</abjad>
\end{comment}

%%% ABJADBOOK START %%%
\begin{singlespacing}
\vspace{-0.5\baselineskip}
\begin{lstlisting}
>>> malformed_timespan = timespantools.Timespan(0, 0)
>>> malformed_timespan.is_well_formed
False
\end{lstlisting}
\end{singlespacing}
%%% ABJADBOOK END %%%

\begin{comment}
<abjad>
templated_timespan = new(timespan, stop_offset=(5, 16))
print(format(templated_timespan))
</abjad>
\end{comment}

%%% ABJADBOOK START %%%
\begin{singlespacing}
\vspace{-0.5\baselineskip}
\begin{lstlisting}
>>> templated_timespan = new(timespan, stop_offset=(5, 16))
>>> print(format(templated_timespan))
timespantools.Timespan(
    start_offset=durationtools.Offset(1, 4),
    stop_offset=durationtools.Offset(5, 16),
    )
\end{lstlisting}
\end{singlespacing}
%%% ABJADBOOK END %%%

\begin{comment}
<abjad>
annotated_timespan = timespantools.AnnotatedTimespan(
    start_offset=(1, 8),
    stop_offset=(7, 8),
    annotation='Any arbitrary object can act as an annotation.'
    )
annotated_timespan.annotation
</abjad>
\end{comment}

%%% ABJADBOOK START %%%
\begin{singlespacing}
\vspace{-0.5\baselineskip}
\begin{lstlisting}
>>> annotated_timespan = timespantools.AnnotatedTimespan(
...     start_offset=(1, 8),
...     stop_offset=(7, 8),
...     annotation='Any arbitrary object can act as an annotation.'
...     )
>>> annotated_timespan.annotation
'Any arbitrary object can act as an annotation.'
\end{lstlisting}
\end{singlespacing}
%%% ABJADBOOK END %%%

\subsection{Time relations}

- time relations: intersection, congruency etc.

\begin{comment}
<abjad>
timespan_1 = timespantools.Timespan(0, 10)
timespan_2 = timespantools.Timespan(5, 15)
timespan_3 = timespantools.Timespan(10, 15)
</abjad>
\end{comment}

%%% ABJADBOOK START %%%
\begin{singlespacing}
\vspace{-0.5\baselineskip}
\begin{lstlisting}
>>> timespan_1 = timespantools.Timespan(0, 10)
>>> timespan_2 = timespantools.Timespan(5, 15)
>>> timespan_3 = timespantools.Timespan(10, 15)
\end{lstlisting}
\end{singlespacing}
%%% ABJADBOOK END %%%

\begin{comment}
<abjad>
timespan_1.intersects_timespan(timespan_2)
timespan_1.intersects_timespan(timespan_3)
timespan_2.intersects_timespan(timespan_1)
timespan_2.intersects_timespan(timespan_3)
timespan_3.intersects_timespan(timespan_1)
timespan_3.intersects_timespan(timespan_2)
</abjad>
\end{comment}

%%% ABJADBOOK START %%%
\begin{singlespacing}
\vspace{-0.5\baselineskip}
\begin{lstlisting}
>>> timespan_1.intersects_timespan(timespan_2)
True
\end{lstlisting}
\begin{lstlisting}
>>> timespan_1.intersects_timespan(timespan_3)
False
\end{lstlisting}
\begin{lstlisting}
>>> timespan_2.intersects_timespan(timespan_1)
True
\end{lstlisting}
\begin{lstlisting}
>>> timespan_2.intersects_timespan(timespan_3)
True
\end{lstlisting}
\begin{lstlisting}
>>> timespan_3.intersects_timespan(timespan_1)
False
\end{lstlisting}
\begin{lstlisting}
>>> timespan_3.intersects_timespan(timespan_2)
True
\end{lstlisting}
\end{singlespacing}
%%% ABJADBOOK END %%%

\begin{comment}
<abjad>
timespan_1.is_congruent_to_timespan(timespan_2)
timespan_1.is_congruent_to_timespan(timespan_1)
</abjad>
\end{comment}

%%% ABJADBOOK START %%%
\begin{singlespacing}
\vspace{-0.5\baselineskip}
\begin{lstlisting}
>>> timespan_1.is_congruent_to_timespan(timespan_2)
False
\end{lstlisting}
\begin{lstlisting}
>>> timespan_1.is_congruent_to_timespan(timespan_1)
True
\end{lstlisting}
\end{singlespacing}
%%% ABJADBOOK END %%%

\begin{comment}
<abjad>
timespan_1.is_tangent_to_timespan(timespan_2)
timespan_1.is_tangent_to_timespan(timespan_3)
</abjad>
\end{comment}

%%% ABJADBOOK START %%%
\begin{singlespacing}
\vspace{-0.5\baselineskip}
\begin{lstlisting}
>>> timespan_1.is_tangent_to_timespan(timespan_2)
False
\end{lstlisting}
\begin{lstlisting}
>>> timespan_1.is_tangent_to_timespan(timespan_3)
True
\end{lstlisting}
\end{singlespacing}
%%% ABJADBOOK END %%%

\subsection{Operations on timespans}

Consider the following three timespans again.

\begin{comment}
<abjad>
timespan_1 = timespantools.Timespan(0, 10)
timespan_2 = timespantools.Timespan(5, 15)
timespan_3 = timespantools.Timespan(10, 15)
</abjad>
\end{comment}

%%% ABJADBOOK START %%%
\begin{singlespacing}
\vspace{-0.5\baselineskip}
\begin{lstlisting}
>>> timespan_1 = timespantools.Timespan(0, 10)
>>> timespan_2 = timespantools.Timespan(5, 15)
>>> timespan_3 = timespantools.Timespan(10, 15)
\end{lstlisting}
\end{singlespacing}
%%% ABJADBOOK END %%%

The logical AND of any two timespans can be computed.

\begin{comment}
<abjad>
timespan_1 & timespan_2
timespan_1 & timespan_3
timespan_2 & timespan_3
</abjad>
\end{comment}

%%% ABJADBOOK START %%%
\begin{singlespacing}
\vspace{-0.5\baselineskip}
\begin{lstlisting}
>>> timespan_1 & timespan_2
TimespanInventory([Timespan(start_offset=Offset(5, 1), stop_offset=Offset(10, 1))])
\end{lstlisting}
\begin{lstlisting}
>>> timespan_1 & timespan_3
TimespanInventory([])
\end{lstlisting}
\begin{lstlisting}
>>> timespan_2 & timespan_3
TimespanInventory([Timespan(start_offset=Offset(10, 1), stop_offset=Offset(15, 1))])
\end{lstlisting}
\end{singlespacing}
%%% ABJADBOOK END %%%

The logical OR of any two timespans can be computed.

\begin{comment}
<abjad>
timespan_1 | timespan_2
timespan_1 | timespan_3
timespan_2 | timespan_3
</abjad>
\end{comment}

%%% ABJADBOOK START %%%
\begin{singlespacing}
\vspace{-0.5\baselineskip}
\begin{lstlisting}
>>> timespan_1 | timespan_2
TimespanInventory([Timespan(start_offset=Offset(0, 1), stop_offset=Offset(15, 1))])
\end{lstlisting}
\begin{lstlisting}
>>> timespan_1 | timespan_3
TimespanInventory([Timespan(start_offset=Offset(0, 1), stop_offset=Offset(15, 1))])
\end{lstlisting}
\begin{lstlisting}
>>> timespan_2 | timespan_3
TimespanInventory([Timespan(start_offset=Offset(5, 1), stop_offset=Offset(15, 1))])
\end{lstlisting}
\end{singlespacing}
%%% ABJADBOOK END %%%

Timespan subtraction is another crucial operation.

\begin{comment}
<abjad>
timespan_1 = timespantools.Timespan(0, 15)
timespan_2 = timespantools.Timespan(5, 10)
timespan_3 = timespantools.Timespan(10, 20)
</abjad>
\end{comment}

%%% ABJADBOOK START %%%
\begin{singlespacing}
\vspace{-0.5\baselineskip}
\begin{lstlisting}
>>> timespan_1 = timespantools.Timespan(0, 15)
>>> timespan_2 = timespantools.Timespan(5, 10)
>>> timespan_3 = timespantools.Timespan(10, 20)
\end{lstlisting}
\end{singlespacing}
%%% ABJADBOOK END %%%

\begin{comment}
<abjad>
print(format(timespan_1 - timespan_1))
print(format(timespan_1 - timespan_2))
print(format(timespan_1 - timespan_3))
print(format(timespan_2 - timespan_1))
print(format(timespan_2 - timespan_2))
print(format(timespan_2 - timespan_3))
print(format(timespan_3 - timespan_1))
print(format(timespan_3 - timespan_2))
print(format(timespan_3 - timespan_3))
</abjad>
\end{comment}

%%% ABJADBOOK START %%%
\begin{singlespacing}
\vspace{-0.5\baselineskip}
\begin{lstlisting}
>>> print(format(timespan_1 - timespan_1))
timespantools.TimespanInventory(
    []
    )
\end{lstlisting}
\begin{lstlisting}
>>> print(format(timespan_1 - timespan_2))
timespantools.TimespanInventory(
    [
        timespantools.Timespan(
            start_offset=durationtools.Offset(0, 1),
            stop_offset=durationtools.Offset(5, 1),
            ),
        timespantools.Timespan(
            start_offset=durationtools.Offset(10, 1),
            stop_offset=durationtools.Offset(15, 1),
            ),
        ]
    )
\end{lstlisting}
\begin{lstlisting}
>>> print(format(timespan_1 - timespan_3))
timespantools.TimespanInventory(
    [
        timespantools.Timespan(
            start_offset=durationtools.Offset(0, 1),
            stop_offset=durationtools.Offset(10, 1),
            ),
        ]
    )
\end{lstlisting}
\begin{lstlisting}
>>> print(format(timespan_2 - timespan_1))
timespantools.TimespanInventory(
    []
    )
\end{lstlisting}
\begin{lstlisting}
>>> print(format(timespan_2 - timespan_2))
timespantools.TimespanInventory(
    []
    )
\end{lstlisting}
\begin{lstlisting}
>>> print(format(timespan_2 - timespan_3))
timespantools.TimespanInventory(
    [
        timespantools.Timespan(
            start_offset=durationtools.Offset(5, 1),
            stop_offset=durationtools.Offset(10, 1),
            ),
        ]
    )
\end{lstlisting}
\begin{lstlisting}
>>> print(format(timespan_3 - timespan_1))
timespantools.TimespanInventory(
    [
        timespantools.Timespan(
            start_offset=durationtools.Offset(15, 1),
            stop_offset=durationtools.Offset(20, 1),
            ),
        ]
    )
\end{lstlisting}
\begin{lstlisting}
>>> print(format(timespan_3 - timespan_2))
timespantools.TimespanInventory(
    [
        timespantools.Timespan(
            start_offset=durationtools.Offset(10, 1),
            stop_offset=durationtools.Offset(20, 1),
            ),
        ]
    )
\end{lstlisting}
\begin{lstlisting}
>>> print(format(timespan_3 - timespan_3))
timespantools.TimespanInventory(
    []
    )
\end{lstlisting}
\end{singlespacing}
%%% ABJADBOOK END %%%

%%%%%%%%%%%%%%%%%%%%%%%%%%%%%%%%%%%%%%%%%%%%%%%%%%%%%%%%%%%%%%%%%%%%%%%%%%%%%%%
\section{Timespan inventories}
%%%%%%%%%%%%%%%%%%%%%%%%%%%%%%%%%%%%%%%%%%%%%%%%%%%%%%%%%%%%%%%%%%%%%%%%%%%%%%%

Timespans can be aggregated together in an instance of the TimespanInventory
class. In addition to the protocol defined for ordered collections, timespan
inventories provide a variety of other methods and properties for working
specifically with timespans.

\begin{comment}
<abjad>
timespan_inventory = timespantools.TimespanInventory([
    timespantools.Timespan(0, 16),
    timespantools.Timespan(5, 12),
    timespantools.Timespan(-2, 8),
    ])
timespan_inventory.timespan
timespan_inventory.duration
timespan_inventory.start_offset
timespan_inventory.stop_offset
timespan_inventory.append(timespantools.Timespan(15, 20))
timespan_inventory.sort()
timespan_inventory.duration
</abjad>
\end{comment}

%%% ABJADBOOK START %%%
\begin{singlespacing}
\vspace{-0.5\baselineskip}
\begin{lstlisting}
>>> timespan_inventory = timespantools.TimespanInventory([
...     timespantools.Timespan(0, 16),
...     timespantools.Timespan(5, 12),
...     timespantools.Timespan(-2, 8),
...     ])
>>> timespan_inventory.timespan
Timespan(start_offset=Offset(-2, 1), stop_offset=Offset(16, 1))
\end{lstlisting}
\begin{lstlisting}
>>> timespan_inventory.duration
Duration(18, 1)
\end{lstlisting}
\begin{lstlisting}
>>> timespan_inventory.start_offset
Offset(-2, 1)
\end{lstlisting}
\begin{lstlisting}
>>> timespan_inventory.stop_offset
Offset(16, 1)
\end{lstlisting}
\begin{lstlisting}
>>> timespan_inventory.append(timespantools.Timespan(15, 20))
>>> timespan_inventory.sort()
>>> timespan_inventory.duration
Duration(22, 1)
\end{lstlisting}
\end{singlespacing}
%%% ABJADBOOK END %%%

\subsection{Inspecting timespan inventories}

\begin{comment}
TimespanInventory.all_are_contiguous
TimespanInventory.all_are_nonoverlapping
\end{comment}

\subsection{Unioning, differencing and splitting}

\begin{comment}
<abjad>
timespan_inventory = timespantools.TimespanInventory([
    timespantools.Timespan(0, 16),
    timespantools.Timespan(5, 12),
    timespantools.Timespan(-2, 8),
    ])
timespan = timespantools.Timespan(5, 10)
result = timespan_inventory & timespan
print(format(timespan_inventory))
</abjad>
\end{comment}

%%% ABJADBOOK START %%%
\begin{singlespacing}
\vspace{-0.5\baselineskip}
\begin{lstlisting}
>>> timespan_inventory = timespantools.TimespanInventory([
...     timespantools.Timespan(0, 16),
...     timespantools.Timespan(5, 12),
...     timespantools.Timespan(-2, 8),
...     ])
>>> timespan = timespantools.Timespan(5, 10)
>>> result = timespan_inventory & timespan
>>> print(format(timespan_inventory))
timespantools.TimespanInventory(
    [
        timespantools.Timespan(
            start_offset=durationtools.Offset(5, 1),
            stop_offset=durationtools.Offset(8, 1),
            ),
        timespantools.Timespan(
            start_offset=durationtools.Offset(5, 1),
            stop_offset=durationtools.Offset(10, 1),
            ),
        timespantools.Timespan(
            start_offset=durationtools.Offset(5, 1),
            stop_offset=durationtools.Offset(10, 1),
            ),
        ]
    )
\end{lstlisting}
\end{singlespacing}
%%% ABJADBOOK END %%%

\begin{comment}
<abjad>
timespan_inventory = timespantools.TimespanInventory([
    timespantools.Timespan(0, 16),
    timespantools.Timespan(5, 12),
    timespantools.Timespan(-2, 8),
    ])
timespan = timespantools.Timespan(5, 10)
result = timespan_inventory - timespan
print(format(timespan_inventory))
</abjad>
\end{comment}

%%% ABJADBOOK START %%%
\begin{singlespacing}
\vspace{-0.5\baselineskip}
\begin{lstlisting}
>>> timespan_inventory = timespantools.TimespanInventory([
...     timespantools.Timespan(0, 16),
...     timespantools.Timespan(5, 12),
...     timespantools.Timespan(-2, 8),
...     ])
>>> timespan = timespantools.Timespan(5, 10)
>>> result = timespan_inventory - timespan
>>> print(format(timespan_inventory))
timespantools.TimespanInventory(
    [
        timespantools.Timespan(
            start_offset=durationtools.Offset(-2, 1),
            stop_offset=durationtools.Offset(5, 1),
            ),
        timespantools.Timespan(
            start_offset=durationtools.Offset(0, 1),
            stop_offset=durationtools.Offset(5, 1),
            ),
        timespantools.Timespan(
            start_offset=durationtools.Offset(10, 1),
            stop_offset=durationtools.Offset(12, 1),
            ),
        timespantools.Timespan(
            start_offset=durationtools.Offset(10, 1),
            stop_offset=durationtools.Offset(16, 1),
            ),
        ]
    )
\end{lstlisting}
\end{singlespacing}
%%% ABJADBOOK END %%%

\begin{comment}
<abjad>
timespan_inventory = timespantools.TimespanInventory([
    timespantools.Timespan(0, 3),
    timespantools.Timespan(3, 6),
    timespantools.Timespan(6, 10),
    ])
left, right = timespan_inventory.split_at_offset(4)
print(format(left))
print(format(right))
</abjad>
\end{comment}

%%% ABJADBOOK START %%%
\begin{singlespacing}
\vspace{-0.5\baselineskip}
\begin{lstlisting}
>>> timespan_inventory = timespantools.TimespanInventory([
...     timespantools.Timespan(0, 3),
...     timespantools.Timespan(3, 6),
...     timespantools.Timespan(6, 10),
...     ])
>>> left, right = timespan_inventory.split_at_offset(4)
>>> print(format(left))
timespantools.TimespanInventory(
    [
        timespantools.Timespan(
            start_offset=durationtools.Offset(0, 1),
            stop_offset=durationtools.Offset(3, 1),
            ),
        timespantools.Timespan(
            start_offset=durationtools.Offset(3, 1),
            stop_offset=durationtools.Offset(4, 1),
            ),
        ]
    )
\end{lstlisting}
\begin{lstlisting}
>>> print(format(right))
timespantools.TimespanInventory(
    [
        timespantools.Timespan(
            start_offset=durationtools.Offset(4, 1),
            stop_offset=durationtools.Offset(6, 1),
            ),
        timespantools.Timespan(
            start_offset=durationtools.Offset(6, 1),
            stop_offset=durationtools.Offset(10, 1),
            ),
        ]
    )
\end{lstlisting}
\end{singlespacing}
%%% ABJADBOOK END %%%

\begin{comment}
Timespan.split_at_offsets()
\end{comment}

\subsection{Timewise partitioning}

\begin{comment}
<abjad>
timespan_inventory = timespantools.TimespanInventory([
    timespantools.Timespan(0, 10),
    timespantools.Timespan(5, 15),
    timespantools.Timespan(15, 20),
    timespantools.Timespan(25, 30),
    ])
</abjad>
\end{comment}

%%% ABJADBOOK START %%%
\begin{singlespacing}
\vspace{-0.5\baselineskip}
\begin{lstlisting}
>>> timespan_inventory = timespantools.TimespanInventory([
...     timespantools.Timespan(0, 10),
...     timespantools.Timespan(5, 15),
...     timespantools.Timespan(15, 20),
...     timespantools.Timespan(25, 30),
...     ])
\end{lstlisting}
\end{singlespacing}
%%% ABJADBOOK END %%%

\begin{comment}
<abjad>
for inventory in timespan_inventory.partition():
    print(format(inventory))

</abjad>
\end{comment}

%%% ABJADBOOK START %%%
\begin{singlespacing}
\vspace{-0.5\baselineskip}
\begin{lstlisting}
>>> for inventory in timespan_inventory.partition():
...     print(format(inventory))
...
timespantools.TimespanInventory(
    [
        timespantools.Timespan(
            start_offset=durationtools.Offset(0, 1),
            stop_offset=durationtools.Offset(10, 1),
            ),
        timespantools.Timespan(
            start_offset=durationtools.Offset(5, 1),
            stop_offset=durationtools.Offset(15, 1),
            ),
        ]
    )
timespantools.TimespanInventory(
    [
        timespantools.Timespan(
            start_offset=durationtools.Offset(15, 1),
            stop_offset=durationtools.Offset(20, 1),
            ),
        ]
    )
timespantools.TimespanInventory(
    [
        timespantools.Timespan(
            start_offset=durationtools.Offset(25, 1),
            stop_offset=durationtools.Offset(30, 1),
            ),
        ]
    )
\end{lstlisting}
\end{singlespacing}
%%% ABJADBOOK END %%%

\begin{comment}
<abjad>
for inventory in timespan_inventory.partition(include_tangent_timespans=True):
    print(format(inventory))

</abjad>
\end{comment}

%%% ABJADBOOK START %%%
\begin{singlespacing}
\vspace{-0.5\baselineskip}
\begin{lstlisting}
>>> for inventory in timespan_inventory.partition(include_tangent_timespans=True):
...     print(format(inventory))
...
timespantools.TimespanInventory(
    [
        timespantools.Timespan(
            start_offset=durationtools.Offset(0, 1),
            stop_offset=durationtools.Offset(10, 1),
            ),
        timespantools.Timespan(
            start_offset=durationtools.Offset(5, 1),
            stop_offset=durationtools.Offset(15, 1),
            ),
        timespantools.Timespan(
            start_offset=durationtools.Offset(15, 1),
            stop_offset=durationtools.Offset(20, 1),
            ),
        ]
    )
timespantools.TimespanInventory(
    [
        timespantools.Timespan(
            start_offset=durationtools.Offset(25, 1),
            stop_offset=durationtools.Offset(30, 1),
            ),
        ]
    )
\end{lstlisting}
\end{singlespacing}
%%% ABJADBOOK END %%%

\subsection{Multiplexing \& demultiplexing}

\subsection{Resolving cascading overlap}

\subsection{Other timespan inventory operations}

\begin{comment}
TimespanInventory.clip_timespan_durations
TimespanInventory.count_offsets()
TimespanInventory.explode()
TimespanInventory.round_offsets()
\end{comment}

\subsection{Optimized timespan inventories}

Consort provides its own timespan collection class -- the TimespanCollection.
This class stores timespans internally not as a list, but in a balanced
\emph{interval tree}\footnote{An interval tree is an augmented self-balancing
binary tree which stores both start offsets as well as stop offsets.}
datastructure which guarantees sorting and allows for highly optimized lookups
of timespans intersecting specific offsets. This class is used at crucial
points during Consort's interpretation stage simply for purposes of speed, and
should be considered an implementation detail. With work, its internal
datastructure will eventually be merged into Abjad's TimespanInventory.

%%%%%%%%%%%%%%%%%%%%%%%%%%%%%%%%%%%%%%%%%%%%%%%%%%%%%%%%%%%%%%%%%%%%%%%%%%%%%%%
\section{Annotated timespans in \emph{Consort}}
%%%%%%%%%%%%%%%%%%%%%%%%%%%%%%%%%%%%%%%%%%%%%%%%%%%%%%%%%%%%%%%%%%%%%%%%%%%%%%%

We need to discuss the products of timespan makers before we can discuss
timespan makers themselves.

\subsection{Payloaded timespans}

- layer

- voice name

\subsection{\emph{Performed} timespans}

\begin{comment}
<abjad>
performed_timespan = consort.PerformedTimespan(
    layer=1,
    minimum_duration=Duration(1, 8),
    music_specifier=consort.MusicSpecifier(),
    start_offset=Offset(1, 4),
    stop_offset=Offset(2, 1),
    voice_name='Violin 1 LH Voice',
    )
</abjad>
\end{comment}

%%% ABJADBOOK START %%%
\begin{singlespacing}
\vspace{-0.5\baselineskip}
\begin{lstlisting}
>>> performed_timespan = consort.PerformedTimespan(
...     layer=1,
...     minimum_duration=Duration(1, 8),
...     music_specifier=consort.MusicSpecifier(),
...     start_offset=Offset(1, 4),
...     stop_offset=Offset(2, 1),
...     voice_name='Violin 1 LH Voice',
...     )
\end{lstlisting}
\end{singlespacing}
%%% ABJADBOOK END %%%

- forbid fusing

- forbid splitting

- minimum duration

- (additionally, music specifier: minimum phrase duration)

- divisions

- music

- music specifier

\subsection{\emph{Silent} timespans}

\begin{comment}
<abjad>
silent_timespan = consort.SilentTimespan(
    layer=2,
    start_offset=Offset(0, 1),
    stop_offset=Offset(1, 4),
    voice_name='Violin 1 LH Voice',
    )
</abjad>
\end{comment}

%%% ABJADBOOK START %%%
\begin{singlespacing}
\vspace{-0.5\baselineskip}
\begin{lstlisting}
>>> silent_timespan = consort.SilentTimespan(
...     layer=2,
...     start_offset=Offset(0, 1),
...     stop_offset=Offset(1, 4),
...     voice_name='Violin 1 LH Voice',
...     )
\end{lstlisting}
\end{singlespacing}
%%% ABJADBOOK END %%%

%%%%%%%%%%%%%%%%%%%%%%%%%%%%%%%%%%%%%%%%%%%%%%%%%%%%%%%%%%%%%%%%%%%%%%%%%%%%%%%
\section{Timespan makers}
%%%%%%%%%%%%%%%%%%%%%%%%%%%%%%%%%%%%%%%%%%%%%%%%%%%%%%%%%%%%%%%%%%%%%%%%%%%%%%%

- timespan specifier

- independent vs dependent

- target timespans

- talea

- padding

\subsection{FloodedTimespanMaker}

\begin{comment}
<abjad>
flooded_timespan_maker = consort.FloodedTimespanMaker()
print(format(flooded_timespan_maker))
</abjad>
\end{comment}

%%% ABJADBOOK START %%%
\begin{singlespacing}
\vspace{-0.5\baselineskip}
\begin{lstlisting}
>>> flooded_timespan_maker = consort.FloodedTimespanMaker()
>>> print(format(flooded_timespan_maker))
consort.tools.FloodedTimespanMaker()
\end{lstlisting}
\end{singlespacing}
%%% ABJADBOOK END %%%

\begin{comment}
<abjad>
music_specifiers = {'Violin Voice': 'violin music'}
target_timespan = timespantools.Timespan((1, 4), (11, 8))
timespan_inventory = flooded_timespan_maker(
    music_specifiers=music_specifiers,
    target_timespan=target_timespan,
    )
print(format(timespan_inventory))
</abjad>
\end{comment}

%%% ABJADBOOK START %%%
\begin{singlespacing}
\vspace{-0.5\baselineskip}
\begin{lstlisting}
>>> music_specifiers = {'Violin Voice': 'violin music'}
>>> target_timespan = timespantools.Timespan((1, 4), (11, 8))
>>> timespan_inventory = flooded_timespan_maker(
...     music_specifiers=music_specifiers,
...     target_timespan=target_timespan,
...     )
>>> print(format(timespan_inventory))
timespantools.TimespanInventory(
    [
        consort.tools.PerformedTimespan(
            music_specifier='violin music',
            start_offset=durationtools.Offset(1, 4),
            stop_offset=durationtools.Offset(11, 8),
            voice_name='Violin Voice',
            ),
        ]
    )
\end{lstlisting}
\end{singlespacing}
%%% ABJADBOOK END %%%

Adding a second music specifier entry and a layer keyword generates another
collection of timespans.

\begin{comment}
<abjad>
music_specifiers = {
    'Violin Voice': 'violin music',
    'Cello Voice': 'cello music',
    }
timespan_inventory = flooded_timespan_maker(
    layer=3,
    music_specifiers=music_specifiers,
    target_timespan=target_timespan,
    )
print(format(timespan_inventory))
</abjad>
\end{comment}

%%% ABJADBOOK START %%%
\begin{singlespacing}
\vspace{-0.5\baselineskip}
\begin{lstlisting}
>>> music_specifiers = {
...     'Violin Voice': 'violin music',
...     'Cello Voice': 'cello music',
...     }
>>> timespan_inventory = flooded_timespan_maker(
...     layer=3,
...     music_specifiers=music_specifiers,
...     target_timespan=target_timespan,
...     )
>>> print(format(timespan_inventory))
timespantools.TimespanInventory(
    [
        consort.tools.PerformedTimespan(
            layer=3,
            music_specifier='cello music',
            start_offset=durationtools.Offset(1, 4),
            stop_offset=durationtools.Offset(11, 8),
            voice_name='Cello Voice',
            ),
        consort.tools.PerformedTimespan(
            layer=3,
            music_specifier='violin music',
            start_offset=durationtools.Offset(1, 4),
            stop_offset=durationtools.Offset(11, 8),
            voice_name='Violin Voice',
            ),
        ]
    )
\end{lstlisting}
\end{singlespacing}
%%% ABJADBOOK END %%%

A new flooded timespan maker, configured with padding and a timespan specifier
which will further configure each generated timespan.

\begin{comment}
<abjad>
flooded_timespan_maker = consort.FloodedTimespanMaker(
    padding=Duration(1, 4),
    timespan_specifier=consort.TimespanSpecifier(
        minimum_duration=Duration(1, 8),
        ),
    )
timespan_inventory = flooded_timespan_maker(
    layer=5,
    music_specifiers=music_specifiers,
    target_timespan=target_timespan,
    )
print(format(timespan_inventory))
</abjad>
\end{comment}

%%% ABJADBOOK START %%%
\begin{singlespacing}
\vspace{-0.5\baselineskip}
\begin{lstlisting}
>>> flooded_timespan_maker = consort.FloodedTimespanMaker(
...     padding=Duration(1, 4),
...     timespan_specifier=consort.TimespanSpecifier(
...         minimum_duration=Duration(1, 8),
...         ),
...     )
>>> timespan_inventory = flooded_timespan_maker(
...     layer=5,
...     music_specifiers=music_specifiers,
...     target_timespan=target_timespan,
...     )
>>> print(format(timespan_inventory))
timespantools.TimespanInventory(
    [
        consort.tools.SilentTimespan(
            start_offset=durationtools.Offset(0, 1),
            stop_offset=durationtools.Offset(1, 4),
            layer=5,
            voice_name='Violin Voice',
            ),
        consort.tools.SilentTimespan(
            start_offset=durationtools.Offset(0, 1),
            stop_offset=durationtools.Offset(1, 4),
            layer=5,
            voice_name='Cello Voice',
            ),
        consort.tools.PerformedTimespan(
            layer=5,
            minimum_duration=durationtools.Duration(1, 8),
            music_specifier='cello music',
            start_offset=durationtools.Offset(1, 4),
            stop_offset=durationtools.Offset(11, 8),
            voice_name='Cello Voice',
            ),
        consort.tools.PerformedTimespan(
            layer=5,
            minimum_duration=durationtools.Duration(1, 8),
            music_specifier='violin music',
            start_offset=durationtools.Offset(1, 4),
            stop_offset=durationtools.Offset(11, 8),
            voice_name='Violin Voice',
            ),
        consort.tools.SilentTimespan(
            start_offset=durationtools.Offset(11, 8),
            stop_offset=durationtools.Offset(13, 8),
            layer=5,
            voice_name='Violin Voice',
            ),
        consort.tools.SilentTimespan(
            start_offset=durationtools.Offset(11, 8),
            stop_offset=durationtools.Offset(13, 8),
            layer=5,
            voice_name='Cello Voice',
            ),
        ]
    )
\end{lstlisting}
\end{singlespacing}
%%% ABJADBOOK END %%%

\subsection{TaleaTimespanMaker}

\begin{comment}
<abjad>
timespan_maker = consort.TaleaTimespanMaker(
    initial_silence_talea=rhythmmakertools.Talea(
        counts=(0, 4),
        denominator=16,
        )
    )
</abjad>

%%% ABJADBOOK START %%%
\begin{singlespacing}
\vspace{-0.5\baselineskip}
\begin{lstlisting}
>>> timespan_maker = consort.TaleaTimespanMaker(
...     initial_silence_talea=rhythmmakertools.Talea(
...         counts=(0, 4),
...         denominator=16,
...         )
...     )
\end{lstlisting}
\end{singlespacing}
%%% ABJADBOOK END %%%

- taleas: playing, silence and initial silence

- groupings

- synchronization

- repeat and reflect

\subsection{DependentTimespanMaker}

\begin{comment}
<abjad>
dependent_timespan_maker = consort.DependentTimespanMaker(
    include_inner_starts=True,
    include_inner_stops=False,
    voice_names=(
        'Piano Upper Voice',
        'Piano Lower Voice',
        )
    )
</abjad>
\end{comment}

%%% ABJADBOOK START %%%
\begin{singlespacing}
\vspace{-0.5\baselineskip}
\begin{lstlisting}
>>> dependent_timespan_maker = consort.DependentTimespanMaker(
...     include_inner_starts=True,
...     include_inner_stops=False,
...     voice_names=(
...         'Piano Upper Voice',
...         'Piano Lower Voice',
...         )
...     )
\end{lstlisting}
\end{singlespacing}
%%% ABJADBOOK END %%%

%%%%%%%%%%%%%%%%%%%%%%%%%%%%%%%%%%%%%%%%%%%%%%%%%%%%%%%%%%%%%%%%%%%%%%%%%%%%%%%
\section{Rhythm makers}
%%%%%%%%%%%%%%%%%%%%%%%%%%%%%%%%%%%%%%%%%%%%%%%%%%%%%%%%%%%%%%%%%%%%%%%%%%%%%%%

- divisions

\begin{comment}
<abjad>
divisions = [(3, 8), (4, 8), (3, 16), (4, 16), (5, 8), (2, 4)]
</abjad>
\end{comment}

%%% ABJADBOOK START %%%
\begin{singlespacing}
\vspace{-0.5\baselineskip}
\begin{lstlisting}
>>> divisions = [(3, 8), (4, 8), (3, 16), (4, 16), (5, 8), (2, 4)]
\end{lstlisting}
\end{singlespacing}
%%% ABJADBOOK END %%%

- rhythm maker

\subsection{Rhythm maker configuration}

- specifiers: tie, duration spelling, beam

\subsection{Examples}

\subsubsection{NoteRhythmMaker}

\begin{comment}
<abjad>
note_rhythm_maker = rhythmmakertools.NoteRhythmMaker(
    )
show(note_rhythm_maker, divisions=divisions)
</abjad>
\end{comment}

%%% ABJADBOOK START %%%
\begin{singlespacing}
\vspace{-0.5\baselineskip}
\begin{lstlisting}
>>> note_rhythm_maker = rhythmmakertools.NoteRhythmMaker(
...     )
>>> show(note_rhythm_maker, divisions=divisions)
\end{lstlisting}
\includegraphics{assets/lilypond-d0f66021e4860194d32ee0fa226ed174.pdf}
\end{singlespacing}
%%% ABJADBOOK END %%%

\subsubsection{EvenDivisionsRhythmMaker}

\begin{comment}
<abjad>
even_division_rhythm_maker = rhythmmakertools.EvenDivisionRhythmMaker(
    denominators=[8, 16, 4],
    )
show(even_division_rhythm_maker, divisions=divisions)
</abjad>
\end{comment}

%%% ABJADBOOK START %%%
\begin{singlespacing}
\vspace{-0.5\baselineskip}
\begin{lstlisting}
>>> even_division_rhythm_maker = rhythmmakertools.EvenDivisionRhythmMaker(
...     denominators=[8, 16, 4],
...     )
>>> show(even_division_rhythm_maker, divisions=divisions)
\end{lstlisting}
\includegraphics{assets/lilypond-06c5f723997363adb1e3e9391d01892a.pdf}
\end{singlespacing}
%%% ABJADBOOK END %%%

\subsubsection{IncisedRhythmMaker}

\begin{comment}
<abjad>
incised_rhythm_maker = rhythmmakertools.IncisedRhythmMaker(
    incise_specifier=rhythmmakertools.InciseSpecifier(
        prefix_counts=[0],
        suffix_talea=[-1],
        suffix_counts=[1],
        talea_denominator=16,
        ),
    )
show(incised_rhythm_maker, divisions=divisions)
</abjad>
\end{comment}

%%% ABJADBOOK START %%%
\begin{singlespacing}
\vspace{-0.5\baselineskip}
\begin{lstlisting}
>>> incised_rhythm_maker = rhythmmakertools.IncisedRhythmMaker(
...     incise_specifier=rhythmmakertools.InciseSpecifier(
...         prefix_counts=[0],
...         suffix_talea=[-1],
...         suffix_counts=[1],
...         talea_denominator=16,
...         ),
...     )
>>> show(incised_rhythm_maker, divisions=divisions)
\end{lstlisting}
\includegraphics{assets/lilypond-f21215b14c3e76d687060a020bdca52a.pdf}
\end{singlespacing}
%%% ABJADBOOK END %%%

\subsubsection{TaleaRhythmMaker}

\begin{comment}
<abjad>
talea_rhythm_maker = rhythmmakertools.TaleaRhythmMaker(
    talea=rhythmmakertools.Talea(
        counts=[1, 2, 3, 4],
        denominator=16,
        ),
    )
show(talea_rhythm_maker, divisions=divisions)
</abjad>
\end{comment}

%%% ABJADBOOK START %%%
\begin{singlespacing}
\vspace{-0.5\baselineskip}
\begin{lstlisting}
>>> talea_rhythm_maker = rhythmmakertools.TaleaRhythmMaker(
...     talea=rhythmmakertools.Talea(
...         counts=[1, 2, 3, 4],
...         denominator=16,
...         ),
...     )
>>> show(talea_rhythm_maker, divisions=divisions)
\end{lstlisting}
\includegraphics{assets/lilypond-101e0079ec09025cbd79b4f02fb37c60.pdf}
\end{singlespacing}
%%% ABJADBOOK END %%%

\subsection{\emph{Consort}'s composite rhythm maker}

\begin{comment}
<abjad>
composite_rhythm_maker = consort.CompositeRhythmMaker(
    default=note_rhythm_maker,
    last=incised_rhythm_maker,
    first=even_division_rhythm_maker,
    )
</abjad>
\end{comment}

%%% ABJADBOOK START %%%
\begin{singlespacing}
\vspace{-0.5\baselineskip}
\begin{lstlisting}
>>> composite_rhythm_maker = consort.CompositeRhythmMaker(
...     default=note_rhythm_maker,
...     last=incised_rhythm_maker,
...     first=even_division_rhythm_maker,
...     )
\end{lstlisting}
\end{singlespacing}
%%% ABJADBOOK END %%%

%%%%%%%%%%%%%%%%%%%%%%%%%%%%%%%%%%%%%%%%%%%%%%%%%%%%%%%%%%%%%%%%%%%%%%%%%%%%%%%
\section{Modeling meter}
%%%%%%%%%%%%%%%%%%%%%%%%%%%%%%%%%%%%%%%%%%%%%%%%%%%%%%%%%%%%%%%%%%%%%%%%%%%%%%%

Abjad models meter as a \emph{rhythm-tree} of nested, durated nodes which
outline a series of strongly and weakly accented offsets. The accent strength
of a particular offset found in a meter's rhythm-tree derives from the number
of nodes in that tree sharing that offset as a start or stop. The more nodes in
the rhythm-tree which share an offset, the greater the weight -- the
accentedness -- of that offset is taken to be. Abjad can construct the rhythm
tree for any meter from a numerator / denominator pair such as a rational
duration or time signature. Meter construction involves the progressive
division of the numerator of the input pair into groups of two and
threes\footnote{The factors 4 and 5 are also used in meter rhythm-tree
generation as they provide better typical results during meter rewriting.}, and
the decomposition of any other prime factors into groups of threes and twos.
Division by two always occurs before division by three, giving preference to
even metrical structures above odd or otherwise prime divisions. Constructing
rhythm-trees in this fashion gives results which generally align with common
practice expectations.

Consider the following 6/8 meter and its graph representation:

\begin{comment}
<abjad>
six_eight_meter = metertools.Meter((6, 8))
graph(six_eight_meter)
</abjad>
\end{comment}

%%% ABJADBOOK START %%%
\begin{singlespacing}
\vspace{-0.5\baselineskip}
\begin{lstlisting}
>>> six_eight_meter = metertools.Meter((6, 8))
>>> graph(six_eight_meter)
\end{lstlisting}
\includegraphics[scale=0.5]{assets/graphviz-6867ac3456c6ee33b782ab9744d80562.pdf}
\end{singlespacing}
%%% ABJADBOOK END %%%

\noindent The triangular and rectangular boxes indicate nodes in the
rhythm-tree itself. Rectangular boxes represent \enquote{beats} -- the leaves
of the rhythm-tree -- while triangular boxes indicate larger metrical
groupings. The ovals at the bottom of the graph indicate -- at their top -- the
start or stop offset of the nodes connected to them from above and -- at their
bottom -- the relative weight of their accent. The final oval on the right
indicates the offset and accent weight of the \enquote{next} downbeat.

The topmost triangle in the above graph represent the \enquote{highest}
metrical grouping in a 6/8 meter. Tracing the leftmost and rightmost arrows
down through the topmost node's children gives the offsets 0 and 3/4: the first
downbeat and next downbeat in a 6/8 meter. Offsets 0 and 3/4 also have the
strongest accent weights as they occur as either the start offset or stop
offset of nodes at three levels of hierarchy in the rhythm tree. At the second
level the 6/8 grouping divides into two 3/8 groupings, following common
practice expectations: metrical groupings tend to subdivide into groups of two
before they subdivide into groups of three\footnote{Consider a 12/8 meter.
Western musicians tend to subdivide twelve into either two groups of six or
four groups of three rather than into three groups of four.}. Both second-level
nodes share the offset of 3/8, which also occurs in the third level, giving 3/8
a weight of two. The third level contains the 1/8 duration beats, grouped by
their parents in the second level into two groups three 1/8 duration nodes. The
offsets 1/8, 1/4, 1/2 and 5/8 are not shared by any nodes except at the lowest
metrical level and therefore all receive an accent weight of one.

\subsection{Examples}

Consider the following examples of meters modeled in Abjad.

A 3/4 meter consists of a top-level 3/4 metrical grouping divided into three
1/4 duration beats:

\begin{comment}
<abjad>
three_four_meter = metertools.Meter((3, 4))
graph(three_four_meter)
</abjad>
\end{comment}

%%% ABJADBOOK START %%%
\begin{singlespacing}
\vspace{-0.5\baselineskip}
\begin{lstlisting}
>>> three_four_meter = metertools.Meter((3, 4))
>>> graph(three_four_meter)
\end{lstlisting}
\includegraphics[scale=0.5]{assets/graphviz-106b9d861244019c8f4508e338f00c09.pdf}
\end{singlespacing}
%%% ABJADBOOK END %%%

\noindent By default, a 7/8 meter subdivides its top-level metrical grouping
into 3/8+2/8+2/8 groupings:

\begin{comment}
<abjad>
seven_eight_meter = metertools.Meter((7, 8))
graph(seven_eight_meter)
</abjad>
\end{comment}

%%% ABJADBOOK START %%%
\begin{singlespacing}
\vspace{-0.5\baselineskip}
\begin{lstlisting}
>>> seven_eight_meter = metertools.Meter((7, 8))
>>> graph(seven_eight_meter)
\end{lstlisting}
\includegraphics[scale=0.5]{assets/graphviz-bbce67419d7d935bfc4536be88d9d06a.pdf}
\end{singlespacing}
%%% ABJADBOOK END %%%

\noindent A 12/8 meter subdivides into four 3/8 duration groupings, each
containining three 1/8 duration beats:

\begin{comment}
<abjad>
twelve_eight_meter = metertools.Meter((12, 8))
graph(twelve_eight_meter)
</abjad>
\end{comment}

%%% ABJADBOOK START %%%
\begin{singlespacing}
\vspace{-0.5\baselineskip}
\begin{lstlisting}
>>> twelve_eight_meter = metertools.Meter((12, 8))
>>> graph(twelve_eight_meter)
\end{lstlisting}
\includegraphics[scale=0.5]{assets/graphviz-a4bd974839ddf0257ed007fe6f491351.pdf}
\end{singlespacing}
%%% ABJADBOOK END %%%

\subsection{Alternate meter representations}

Abjad permits alternate representations of meters with the same numerator and
denominator. The default interpretation of 4/4 generates a top-level rhythmic
grouping with a duration of 4/4 and four 1/4 beats as children\footnote{A
\enquote{flat} 4/4 metrical structure is useful for meter rewriting as it
allows the meter rewriting algorithm to ignore many common rhythmic idioms like
1/4+1/2+1/4 and 1/4+3/4.}.

\begin{comment}
<abjad>
four_four_meter = metertools.Meter((4, 4))
graph(four_four_meter)
</abjad>
\end{comment}

%%% ABJADBOOK START %%%
\begin{singlespacing}
\vspace{-0.5\baselineskip}
\begin{lstlisting}
>>> four_four_meter = metertools.Meter((4, 4))
>>> graph(four_four_meter)
\end{lstlisting}
\includegraphics[scale=0.5]{assets/graphviz-08a7fcb5e62e38947325ecc3e2f20393.pdf}
\end{singlespacing}
%%% ABJADBOOK END %%%

\noindent While meter objects are usually instantiated from numerator /
denominator pairs, with their rhythm-tree structure determined programmatically
from that input pair, they can also be instantiated from strings parseable as
rhythm-trees, or from RhythmTree objects themselves. All meters, because they
are implemented in terms of rhythm-trees, can be represented by a Lisp-like
rhythm-tree syntax:

\begin{comment}
<abjad>
print(four_four_meter.pretty_rtm_format)
</abjad>
\end{comment}

%%% ABJADBOOK START %%%
\begin{singlespacing}
\vspace{-0.5\baselineskip}
\begin{lstlisting}
>>> print(four_four_meter.pretty_rtm_format)
(4/4 (
	1/4
	1/4
	1/4
	1/4))
\end{lstlisting}
\end{singlespacing}
%%% ABJADBOOK END %%%

Instantiating meters from explicit rhythm tree syntax allows composers to
choose alternate representations of metrical structures. For example, a 4/4
meter which strongly emphasizes beat three is possible by subdividing the
top-level 4/4 metrical grouping into two 2/4 duration groupings, which are then
subdivided each into two 1/4 duration beats:

\begin{comment}
<abjad>
arbitrary_meter_1 = metertools.Meter('(4/4 ((2/4 (1/4 1/4)) (2/4 (1/4 1/4))))')
graph(arbitrary_meter_1)
</abjad>
\end{comment}

%%% ABJADBOOK START %%%
\begin{singlespacing}
\vspace{-0.5\baselineskip}
\begin{lstlisting}
>>> arbitrary_meter_1 = metertools.Meter('(4/4 ((2/4 (1/4 1/4)) (2/4 (1/4 1/4))))')
>>> graph(arbitrary_meter_1)
\end{lstlisting}
\includegraphics[scale=0.5]{assets/graphviz-70be4a3bd53912533a67115b8f59e873.pdf}
\end{singlespacing}
%%% ABJADBOOK END %%%

\noindent Unusual metrical structures are also possible, such as the following
4/4 meter which divides into two parts, with the first part dividing into two
again, and the second grouping of that divided into two again:

\begin{comment}
<abjad>
arbitrary_meter_2 = metertools.Meter('(4/4 ((2/4 (1/4 (1/4 (1/8 1/8)))) 1/2))')
graph(arbitrary_meter_2)
</abjad>
\end{comment}

%%% ABJADBOOK START %%%
\begin{singlespacing}
\vspace{-0.5\baselineskip}
\begin{lstlisting}
>>> arbitrary_meter_2 = metertools.Meter('(4/4 ((2/4 (1/4 (1/4 (1/8 1/8)))) 1/2))')
>>> graph(arbitrary_meter_2)
\end{lstlisting}
\includegraphics[scale=0.5]{assets/graphviz-22af4041cd4228849368d4ee27d55fbb.pdf}
\end{singlespacing}
%%% ABJADBOOK END %%%

%%%%%%%%%%%%%%%%%%%%%%%%%%%%%%%%%%%%%%%%%%%%%%%%%%%%%%%%%%%%%%%%%%%%%%%%%%%%%%%
\section{Rewriting meters}
%%%%%%%%%%%%%%%%%%%%%%%%%%%%%%%%%%%%%%%%%%%%%%%%%%%%%%%%%%%%%%%%%%%%%%%%%%%%%%%

Notated rhythms can be expressed in multiple ways while maintaining the same
attack-point and duration structure.

\emph{Meter rewriting} formalizes the process of re-notating a rhythm according
to the offset structure inherent to some meter.

Consider the following rhythm:

\begin{comment}
<abjad>
parseable = "abj: | 2/4 c'2 ~ |"
parseable += "| 4/4 c'32 d'2.. ~ d'16 e'32 ~ |"
parseable += "| 2/4 e'2 |"
staff = Staff(parseable)
show(staff)
</abjad>
\end{comment}

%%% ABJADBOOK START %%%
\begin{singlespacing}
\vspace{-0.5\baselineskip}
\begin{lstlisting}
>>> parseable = "abj: | 2/4 c'2 ~ |"
>>> parseable += "| 4/4 c'32 d'2.. ~ d'16 e'32 ~ |"
>>> parseable += "| 2/4 e'2 |"
>>> staff = Staff(parseable)
>>> show(staff)
\end{lstlisting}
\includegraphics{assets/lilypond-329490b244b088f012cb5146859cdab0.pdf}
\end{singlespacing}
%%% ABJADBOOK END %%%

The middle measure is notated in a perfectly valid manner. However, the
double-dotted D does not align with or break against any of the offsets of a
4/4 metrical structure: 0/4, 1/4, 2/4, 3/4 or 4/4.

\begin{comment}
<abjad>
four_four_meter = metertools.Meter((4, 4))
graph(four_four_meter)
</abjad>
\end{comment}

%%% ABJADBOOK START %%%
\begin{singlespacing}
\vspace{-0.5\baselineskip}
\begin{lstlisting}
>>> four_four_meter = metertools.Meter((4, 4))
>>> graph(four_four_meter)
\end{lstlisting}
\includegraphics[scale=0.5]{assets/graphviz-08a7fcb5e62e38947325ecc3e2f20393.pdf}
\end{singlespacing}
%%% ABJADBOOK END %%%

\begin{comment}
<abjad>
mutate(staff[1][:]).rewrite_meter(four_four_meter)
show(staff)
</abjad>
\end{comment}

%%% ABJADBOOK START %%%
\begin{singlespacing}
\vspace{-0.5\baselineskip}
\begin{lstlisting}
>>> mutate(staff[1][:]).rewrite_meter(four_four_meter)
>>> show(staff)
\end{lstlisting}
\includegraphics{assets/lilypond-52ed3c73b623b7b0d9a2c6230084a264.pdf}
\end{singlespacing}
%%% ABJADBOOK END %%%

\begin{comment}
<abjad>
two_two_meter = metertools.Meter((2, 2))
staff = Staff(parseable)
mutate(staff[1][:]).rewrite_meter(two_two_meter)
show(staff)
</abjad>
\end{comment}

%%% ABJADBOOK START %%%
\begin{singlespacing}
\vspace{-0.5\baselineskip}
\begin{lstlisting}
>>> two_two_meter = metertools.Meter((2, 2))
>>> staff = Staff(parseable)
>>> mutate(staff[1][:]).rewrite_meter(two_two_meter)
>>> show(staff)
\end{lstlisting}
\includegraphics{assets/lilypond-467c7a453dd4de93da0cc1c7acbe0a87.pdf}
\end{singlespacing}
%%% ABJADBOOK END %%%

\subsection{Dot count}

\begin{comment}
<abjad>
measure = Measure((3, 4), "c'32 d'8 e'8 fs'4...")
show(measure)
</abjad>
\end{comment}

%%% ABJADBOOK START %%%
\begin{singlespacing}
\vspace{-0.5\baselineskip}
\begin{lstlisting}
>>> measure = Measure((3, 4), "c'32 d'8 e'8 fs'4...")
>>> show(measure)
\end{lstlisting}
\includegraphics{assets/lilypond-8d4bc784b86bdade27d371532ab358a2.pdf}
\end{singlespacing}
%%% ABJADBOOK END %%%

\begin{comment}
<abjad>
mutate(measure[:]).rewrite_meter((3, 4))
show(measure)
</abjad>
\end{comment}

%%% ABJADBOOK START %%%
\begin{singlespacing}
\vspace{-0.5\baselineskip}
\begin{lstlisting}
>>> mutate(measure[:]).rewrite_meter((3, 4))
>>> show(measure)
\end{lstlisting}
\includegraphics{assets/lilypond-d4d52cf91b08080c4067643d24faaf2a.pdf}
\end{singlespacing}
%%% ABJADBOOK END %%%

\begin{comment}
<abjad>
measure = Measure((3, 4), "c'32 d'8 e'8 fs'4...")
mutate(measure[:]).rewrite_meter((3, 4), maximum_dot_count=2)
show(measure)
</abjad>
\end{comment}

%%% ABJADBOOK START %%%
\begin{singlespacing}
\vspace{-0.5\baselineskip}
\begin{lstlisting}
>>> measure = Measure((3, 4), "c'32 d'8 e'8 fs'4...")
>>> mutate(measure[:]).rewrite_meter((3, 4), maximum_dot_count=2)
>>> show(measure)
\end{lstlisting}
\includegraphics{assets/lilypond-40b68312fd44f949b40de163b21588c3.pdf}
\end{singlespacing}
%%% ABJADBOOK END %%%

\begin{comment}
<abjad>
measure = Measure((3, 4), "c'32 d'8 e'8 fs'4...")
mutate(measure[:]).rewrite_meter((3, 4), maximum_dot_count=1)
show(measure)
</abjad>
\end{comment}

%%% ABJADBOOK START %%%
\begin{singlespacing}
\vspace{-0.5\baselineskip}
\begin{lstlisting}
>>> measure = Measure((3, 4), "c'32 d'8 e'8 fs'4...")
>>> mutate(measure[:]).rewrite_meter((3, 4), maximum_dot_count=1)
>>> show(measure)
\end{lstlisting}
\includegraphics{assets/lilypond-5fba6e827130a27fefae4843d2b21248.pdf}
\end{singlespacing}
%%% ABJADBOOK END %%%

\begin{comment}
<abjad>
measure = Measure((3, 4), "c'32 d'8 e'8 fs'4...")
mutate(measure[:]).rewrite_meter((3, 4), maximum_dot_count=0)
show(measure)
</abjad>
\end{comment}

%%% ABJADBOOK START %%%
\begin{singlespacing}
\vspace{-0.5\baselineskip}
\begin{lstlisting}
>>> measure = Measure((3, 4), "c'32 d'8 e'8 fs'4...")
>>> mutate(measure[:]).rewrite_meter((3, 4), maximum_dot_count=0)
>>> show(measure)
\end{lstlisting}
\includegraphics{assets/lilypond-2ca50ee6f27ae56bae476fb9dc1faafa.pdf}
\end{singlespacing}
%%% ABJADBOOK END %%%

\subsection{Boundary depth}

\begin{comment}
<abjad>
nine_eight_meter = metertools.Meter((9, 8))
graph(nine_eight_meter)
</abjad>
\end{comment}

%%% ABJADBOOK START %%%
\begin{singlespacing}
\vspace{-0.5\baselineskip}
\begin{lstlisting}
>>> nine_eight_meter = metertools.Meter((9, 8))
>>> graph(nine_eight_meter)
\end{lstlisting}
\includegraphics[scale=0.5]{assets/graphviz-63c84e9823eef4be1a2ac91cc7cf5a3f.pdf}
\end{singlespacing}
%%% ABJADBOOK END %%%

\begin{comment}
<abjad>
measure = Measure((9, 8), "c'2 d'2 e'8")
show(measure)
</abjad>
\end{comment}

%%% ABJADBOOK START %%%
\begin{singlespacing}
\vspace{-0.5\baselineskip}
\begin{lstlisting}
>>> measure = Measure((9, 8), "c'2 d'2 e'8")
>>> show(measure)
\end{lstlisting}
\includegraphics{assets/lilypond-f459f8c64dd96ba762e4bdebba6f4680.pdf}
\end{singlespacing}
%%% ABJADBOOK END %%%

\begin{comment}
<abjad>
mutate(measure[:]).rewrite_meter(nine_eight_meter)
show(measure)
</abjad>
\end{comment}

%%% ABJADBOOK START %%%
\begin{singlespacing}
\vspace{-0.5\baselineskip}
\begin{lstlisting}
>>> mutate(measure[:]).rewrite_meter(nine_eight_meter)
>>> show(measure)
\end{lstlisting}
\includegraphics{assets/lilypond-63f30ee6e2e244440af9d6f1e0f84a02.pdf}
\end{singlespacing}
%%% ABJADBOOK END %%%

\begin{comment}
<abjad>
measure = Measure((9, 8), "c'2 d'2 e'8")
mutate(measure[:]).rewrite_meter(
    nine_eight_meter,
    boundary_depth=1,
    )
show(measure)
</abjad>
\end{comment}

%%% ABJADBOOK START %%%
\begin{singlespacing}
\vspace{-0.5\baselineskip}
\begin{lstlisting}
>>> measure = Measure((9, 8), "c'2 d'2 e'8")
>>> mutate(measure[:]).rewrite_meter(
...     nine_eight_meter,
...     boundary_depth=1,
...     )
>>> show(measure)
\end{lstlisting}
\includegraphics{assets/lilypond-93c0260083a9ae2ad25a3c7c064e3024.pdf}
\end{singlespacing}
%%% ABJADBOOK END %%%

\subsection{Disambiguating 3/4 \& 6/8}

\begin{comment}
<abjad>
staff = Staff(context_name='RhythmicStaff')
staff.extend("{ c'2 c'4 } { c'4. c'4. } { c'2 ~ c'8 c'8 }")
attach(TimeSignature((3, 4)), staff)
show(staff)
</abjad>
\end{comment}

%%% ABJADBOOK START %%%
\begin{singlespacing}
\vspace{-0.5\baselineskip}
\begin{lstlisting}
>>> staff = Staff(context_name='RhythmicStaff')
>>> staff.extend("{ c'2 c'4 } { c'4. c'4. } { c'2 ~ c'8 c'8 }")
>>> attach(TimeSignature((3, 4)), staff)
>>> show(staff)
\end{lstlisting}
\includegraphics{assets/lilypond-b4a2116ab0031b6068b1112932629bac.pdf}
\end{singlespacing}
%%% ABJADBOOK END %%%

\begin{comment}
<abjad>
for container in staff:
    mutate(container[:]).rewrite_meter((3, 4), boundary_depth=1)

show(staff)
</abjad>
\end{comment}

%%% ABJADBOOK START %%%
\begin{singlespacing}
\vspace{-0.5\baselineskip}
\begin{lstlisting}
>>> for container in staff:
...     mutate(container[:]).rewrite_meter((3, 4), boundary_depth=1)
...
>>> show(staff)
\end{lstlisting}
\includegraphics{assets/lilypond-0cef77f45f31e31ad271d15c9c32edf8.pdf}
\end{singlespacing}
%%% ABJADBOOK END %%%

\begin{comment}
<abjad>
staff = Staff(context_name='RhythmicStaff')
staff.extend("{ c'2 c'4 } { c'4. c'4. } { c'2 ~ c'8 c'8 }")
attach(TimeSignature((6, 8)), staff)
for container in staff:
    mutate(container[:]).rewrite_meter((6, 8), boundary_depth=1)

show(staff)
</abjad>
\end{comment}

%%% ABJADBOOK START %%%
\begin{singlespacing}
\vspace{-0.5\baselineskip}
\begin{lstlisting}
>>> staff = Staff(context_name='RhythmicStaff')
>>> staff.extend("{ c'2 c'4 } { c'4. c'4. } { c'2 ~ c'8 c'8 }")
>>> attach(TimeSignature((6, 8)), staff)
>>> for container in staff:
...     mutate(container[:]).rewrite_meter((6, 8), boundary_depth=1)
...
>>> show(staff)
\end{lstlisting}
\includegraphics{assets/lilypond-58a85c4734fd2a15643aa2409cbff0e5.pdf}
\end{singlespacing}
%%% ABJADBOOK END %%%

\subsection{Recursive meter rewriting}

\begin{comment}
<abjad>
parseable = "abj: | 4/4 c'16 ~ c'4 d'8. ~ "
parseable += "2/3 { d'8. ~ 3/5 { d'16 e'8 ~ e'16 f'16 ~ } } "
parseable += "f'4 |"
measure = parse(parseable)
show(measure)
</abjad>
\end{comment}

%%% ABJADBOOK START %%%
\begin{singlespacing}
\vspace{-0.5\baselineskip}
\begin{lstlisting}
>>> parseable = "abj: | 4/4 c'16 ~ c'4 d'8. ~ "
>>> parseable += "2/3 { d'8. ~ 3/5 { d'16 e'8 ~ e'16 f'16 ~ } } "
>>> parseable += "f'4 |"
>>> measure = parse(parseable)
>>> show(measure)
\end{lstlisting}
\includegraphics{assets/lilypond-270f471c0052391c9bf1a5df2ccc4474.pdf}
\end{singlespacing}
%%% ABJADBOOK END %%%

\begin{comment}
<abjad>
mutate(measure[:]).rewrite_meter(
    measure,
    boundary_depth=1,
    )
show(measure)
</abjad>
\end{comment}

%%% ABJADBOOK START %%%
\begin{singlespacing}
\vspace{-0.5\baselineskip}
\begin{lstlisting}
>>> mutate(measure[:]).rewrite_meter(
...     measure,
...     boundary_depth=1,
...     )
>>> show(measure)
\end{lstlisting}
\includegraphics{assets/lilypond-f7aa660690e64db249ef36ea1eac5ace.pdf}
\end{singlespacing}
%%% ABJADBOOK END %%%

%%%%%%%%%%%%%%%%%%%%%%%%%%%%%%%%%%%%%%%%%%%%%%%%%%%%%%%%%%%%%%%%%%%%%%%%%%%%%%%
\section{Finding meters}
%%%%%%%%%%%%%%%%%%%%%%%%%%%%%%%%%%%%%%%%%%%%%%%%%%%%%%%%%%%%%%%%%%%%%%%%%%%%%%%

A meter's weighted-offset pattern can be used as 1-dimensional kernel, or
convolution matrix, to determine how strongly an arbitrary collection of
offsets appears to express that meter.

Abjad's \texttt{metertools} provides a \texttt{MetricAccentKernel} class.

\subsection{Offset counters}

Before convolving a meter with a collection of offsets, those offsets need to
be extracted. The \texttt{MetricAccentKernel} class'
\texttt{count\_offsets\_in\_expr()} method collects offsets in its input
expression into a \emph{counter} -- a mapping of offsets to the number of times
those offsets appear. Offsets which appear multiple times in the input
expression will result in a higher count in the generated offset counter, and
will in turn have a greater influence during the meter convolution process.

Consider the following score example:

\begin{comment}
<abjad>
upper_staff = Staff("c'8 d'4. e'8 f'4.")
lower_staff = Staff(r'\clef bass c4 b,4 a,2')
piano_staff = scoretools.StaffGroup(
    [upper_staff, lower_staff],
    context_name='PianoStaff',
    )
show(piano_staff)
</abjad>
\end{comment}

%%% ABJADBOOK START %%%
\begin{singlespacing}
\vspace{-0.5\baselineskip}
\begin{lstlisting}
>>> upper_staff = Staff("c'8 d'4. e'8 f'4.")
>>> lower_staff = Staff(r'\clef bass c4 b,4 a,2')
>>> piano_staff = scoretools.StaffGroup(
...     [upper_staff, lower_staff],
...     context_name='PianoStaff',
...     )
>>> show(piano_staff)
\end{lstlisting}
\includegraphics{assets/lilypond-bf3a249e2b4eac93b1884bab1a122a10.pdf}
\end{singlespacing}
%%% ABJADBOOK END %%%

\noindent The offsets found in all of the leaves of the score can be counted by
selecting the score's leaves and passing that selection to
\texttt{count\_offsets\_in\_expr()}. Note that the offset 1/2 shows a count of
4. This is because 1/2 acts as both the start and stop offset for four separate
leaves in the score.

\begin{comment}
<abjad>
leaves = piano_staff.select_leaves(allow_discontiguous_leaves=True)
piano_staff_counter = metertools.MetricAccentKernel.count_offsets_in_expr(
    leaves)
for offset, count in sorted(piano_staff_counter.items()):
    offset, count

</abjad>
\end{comment}

%%% ABJADBOOK START %%%
\begin{singlespacing}
\vspace{-0.5\baselineskip}
\begin{lstlisting}
>>> leaves = piano_staff.select_leaves(allow_discontiguous_leaves=True)
>>> piano_staff_counter = metertools.MetricAccentKernel.count_offsets_in_expr(
...     leaves)
>>> for offset, count in sorted(piano_staff_counter.items()):
...     offset, count
...
(Offset(0, 1), 2)
(Offset(1, 8), 2)
(Offset(1, 4), 2)
(Offset(1, 2), 4)
(Offset(5, 8), 2)
(Offset(1, 1), 2)
\end{lstlisting}
\end{singlespacing}
%%% ABJADBOOK END %%%

Offset counters can also be generated from timespan inventories, allowing meter
convolution to be used without reference to any score objects at all:

\begin{comment}
<abjad>
timespans = timespantools.TimespanInventory([
    timespantools.Timespan(-1, 10),
    timespantools.Timespan(5, 15),
    timespantools.Timespan(15, 20),
    timespantools.Timespan(10, 15),
    ])
timespan_counter = metertools.MetricAccentKernel.count_offsets_in_expr(
    timespans)
for offset, count in sorted(timespan_counter.items()):
    offset, count

</abjad>
\end{comment}

%%% ABJADBOOK START %%%
\begin{singlespacing}
\vspace{-0.5\baselineskip}
\begin{lstlisting}
>>> timespans = timespantools.TimespanInventory([
...     timespantools.Timespan(-1, 10),
...     timespantools.Timespan(5, 15),
...     timespantools.Timespan(15, 20),
...     timespantools.Timespan(10, 15),
...     ])
>>> timespan_counter = metertools.MetricAccentKernel.count_offsets_in_expr(
...     timespans)
>>> for offset, count in sorted(timespan_counter.items()):
...     offset, count
...
(Offset(-1, 1), 1)
(Offset(5, 1), 1)
(Offset(10, 1), 2)
(Offset(15, 1), 3)
(Offset(20, 1), 1)
\end{lstlisting}
\end{singlespacing}
%%% ABJADBOOK END %%%

\subsection{Generating metric accent kernels}

As demonstrated earlier, meters describe a sequence of offsets with varying
degrees of weight, or accentedness, attributed to each offset. Downbeats have a
stronger accent than upbeats, the half-way beat of a 6/8 measure is less strong
than the downbeat but stronger than any of the others, etc.

\begin{comment}
<abjad>
meter = metertools.Meter((4, 4))
kernel_44 = metertools.MetricAccentKernel.from_meter(meter, denominator=8)
for offset, weight in sorted(kernel_44.kernel.items()):
    print('{!s}\t{!s}'.format(offset, weight))

</abjad>
\end{comment}

%%% ABJADBOOK START %%%
\begin{singlespacing}
\vspace{-0.5\baselineskip}
\begin{lstlisting}
>>> meter = metertools.Meter((4, 4))
>>> kernel_44 = metertools.MetricAccentKernel.from_meter(meter, denominator=8)
>>> for offset, weight in sorted(kernel_44.kernel.items()):
...     print('{!s}\t{!s}'.format(offset, weight))
...
0	3/16
1/8	1/16
1/4	1/8
3/8	1/16
1/2	1/8
5/8	1/16
3/4	1/8
7/8	1/16
1	3/16
\end{lstlisting}
\end{singlespacing}
%%% ABJADBOOK END %%%

\subsection{Evaluating metric accent kernels}

\noindent The 4/4 metric accent kernel can be called against an offset counter
-- as though it were a function -- to generate a response via a kind of simple
offset-wise convolution. The count at each offset in the input offset counter
is multiplied against the weight at the corresponding offset in the metric
accent kernel. If no corresponding offset exists in the kernel, the weight is
taken as 0. The weighted counts are then added together and returned.

\begin{comment}
<abjad>
response = kernel_44(piano_staff_counter)
float(response)
</abjad>
\end{comment}

%%% ABJADBOOK START %%%
\begin{singlespacing}
\vspace{-0.5\baselineskip}
\begin{lstlisting}
>>> response = kernel_44(piano_staff_counter)
>>> float(response)
1.75
\end{lstlisting}
\end{singlespacing}
%%% ABJADBOOK END %%%

\begin{comment}
<abjad>
total = Multiplier(0, 1)
for offset, count in sorted(piano_staff_counter.items()):
    weight = Multiplier(0, 1)
    if offset in kernel_44.kernel:
        weight = kernel_44.kernel[offset]
    weighted_count = weight * count
    total += weighted_count
    print(offset, count, weight, weighted_count, total)

</abjad>
\end{comment}

%%% ABJADBOOK START %%%
\begin{singlespacing}
\vspace{-0.5\baselineskip}
\begin{lstlisting}
>>> total = Multiplier(0, 1)
>>> for offset, count in sorted(piano_staff_counter.items()):
...     weight = Multiplier(0, 1)
...     if offset in kernel_44.kernel:
...         weight = kernel_44.kernel[offset]
...     weighted_count = weight * count
...     total += weighted_count
...     print(offset, count, weight, weighted_count, total)
...
(Offset(0, 1), 2, Multiplier(3, 16), Multiplier(3, 8), Multiplier(3, 8))
(Offset(1, 8), 2, Multiplier(1, 16), Multiplier(1, 8), Multiplier(1, 2))
(Offset(1, 4), 2, Multiplier(1, 8), Multiplier(1, 4), Multiplier(3, 4))
(Offset(1, 2), 4, Multiplier(1, 8), Multiplier(1, 2), Multiplier(5, 4))
(Offset(5, 8), 2, Multiplier(1, 16), Multiplier(1, 8), Multiplier(11, 8))
(Offset(1, 1), 2, Multiplier(3, 16), Multiplier(3, 8), Multiplier(7, 4))
\end{lstlisting}
\end{singlespacing}
%%% ABJADBOOK END %%%

\noindent Now consider the metric accent kernels for 3/4, 7/8 and 5/4 meters:

\begin{comment}
<abjad>
kernel_34 = metertools.MetricAccentKernel.from_meter((3, 4), denominator=8)
kernel_78 = metertools.MetricAccentKernel.from_meter((7, 8), denominator=8)
kernel_54 = metertools.MetricAccentKernel.from_meter((5, 4), denominator=8)
</abjad>
\end{comment}

%%% ABJADBOOK START %%%
\begin{singlespacing}
\vspace{-0.5\baselineskip}
\begin{lstlisting}
>>> kernel_34 = metertools.MetricAccentKernel.from_meter((3, 4), denominator=8)
>>> kernel_78 = metertools.MetricAccentKernel.from_meter((7, 8), denominator=8)
>>> kernel_54 = metertools.MetricAccentKernel.from_meter((5, 4), denominator=8)
\end{lstlisting}
\end{singlespacing}
%%% ABJADBOOK END %%%

\noindent We can generate an response for each of these kernels against the
piano staff offset counter.

\begin{comment}
<abjad>
float(kernel_34(piano_staff_counter))
float(kernel_78(piano_staff_counter))
float(kernel_54(piano_staff_counter))
</abjad>
\end{comment}

%%% ABJADBOOK START %%%
\begin{singlespacing}
\vspace{-0.5\baselineskip}
\begin{lstlisting}
>>> float(kernel_34(piano_staff_counter))
1.6923076923076923
\end{lstlisting}
\begin{lstlisting}
>>> float(kernel_78(piano_staff_counter))
1.2857142857142858
\end{lstlisting}
\begin{lstlisting}
>>> float(kernel_54(piano_staff_counter))
1.368421052631579
\end{lstlisting}
\end{singlespacing}
%%% ABJADBOOK END %%%

\noindent Note that the previously recorded response for a 4/4 meter is still
higher than any of these.

\subsection{Meter fitting}

\begin{comment}
<abjad>
permitted_meters = [metertools.Meter(_) for _ in [(3, 4), (4, 4), (5, 4)]]
offsets = [(0, 4), (4, 4), (8, 4), (12, 4), (16, 4)]
for x in metertools.Meter.fit_meters_to_expr(offsets, permitted_meters):
    print(x.implied_time_signature)

</abjad>
\end{comment}

%%% ABJADBOOK START %%%
\begin{singlespacing}
\vspace{-0.5\baselineskip}
\begin{lstlisting}
>>> permitted_meters = [metertools.Meter(_) for _ in [(3, 4), (4, 4), (5, 4)]]
>>> offsets = [(0, 4), (4, 4), (8, 4), (12, 4), (16, 4)]
>>> for x in metertools.Meter.fit_meters_to_expr(offsets, permitted_meters):
...     print(x.implied_time_signature)
...
4/4
4/4
4/4
4/4
\end{lstlisting}
\end{singlespacing}
%%% ABJADBOOK END %%%

\begin{comment}
<abjad>
offsets = [(0, 4), (3, 4), (5, 4), (10, 4), (15, 4), (20, 4)]
for x in metertools.Meter.fit_meters_to_expr(offsets, permitted_meters):
    print(x.implied_time_signature)

</abjad>
\end{comment}

%%% ABJADBOOK START %%%
\begin{singlespacing}
\vspace{-0.5\baselineskip}
\begin{lstlisting}
>>> offsets = [(0, 4), (3, 4), (5, 4), (10, 4), (15, 4), (20, 4)]
>>> for x in metertools.Meter.fit_meters_to_expr(offsets, permitted_meters):
...     print(x.implied_time_signature)
...
3/4
4/4
3/4
5/4
5/4
\end{lstlisting}
\end{singlespacing}
%%% ABJADBOOK END %%%