\chapter{Tools for modeling time, rhythm and meter}

\begin{comment}
<abjad>[hide=true]
import consort
</abjad>
\end{comment}

Consort's implementation of a model of composition relies on a number of
different but interrelated models of musical time.

Dichotomies: outside and inside the score hierarchy, with or without regard to
notation, "coarse" versus "fine" or "phrase" versus "event", vertical or
horizontal, metered and unmetered, potentially simultaneous or strictly
contiguous.

Timespans provide a coarse model of musical time, both in and outside of score
hierarchy.

Notated rhythm provides a fine model of musical time, from within score
hierarchy.

Meter coordinates time and rhythm vertically across score hierarchy, and
bridges the coarse and fine stages of rhythmic interpretation.

Meter is generated as a by-product of phrase-level composition. It is not
specified by-hand during composition. This is not out of any desire to valorize
automaticism, but simply because lacking any other compelling reason to
generate a series of meters I felt the best way for myself would be to have
those meters derive from some sort of pre-existent structure in my
compositional process.

A discussion of these time models and their implications will clarify a later
analysis of the implementation of Consort's score interpretation
stage.

\section{timespans and timespan inventories}

\subsection{anatomy of a timespan}

\begin{comment}
<abjad>
timespan = timespantools.Timespan(
    start_offset=Offset(1, 4),
    stop_offset=Offset(3, 2),
    )
</abjad>
\end{comment}

%%% ABJADBOOK START %%%
\begin{singlespacing}
\begin{lstlisting}
>>> timespan = timespantools.Timespan(
...     start_offset=Offset(1, 4),
...     stop_offset=Offset(3, 2),
...     )
\end{lstlisting}
\end{singlespacing}
%%% ABJADBOOK END %%%

\begin{comment}
<abjad>
timespan.start_offset
timespan.stop_offset
timespan.duration
timespan.is_well_formed
</abjad>
\end{comment}

%%% ABJADBOOK START %%%
\begin{singlespacing}
\begin{lstlisting}
>>> timespan.start_offset
Offset(1, 4)
\end{lstlisting}
\begin{lstlisting}
>>> timespan.stop_offset
Offset(3, 2)
\end{lstlisting}
\begin{lstlisting}
>>> timespan.duration
Duration(5, 4)
\end{lstlisting}
\begin{lstlisting}
>>> timespan.is_well_formed
True
\end{lstlisting}
\end{singlespacing}
%%% ABJADBOOK END %%%

\begin{comment}
<abjad>
malformed_timespan = timespantools.Timespan(0, 0)
malformed_timespan.is_well_formed
</abjad>
\end{comment}

%%% ABJADBOOK START %%%
\begin{singlespacing}
\begin{lstlisting}
>>> malformed_timespan = timespantools.Timespan(0, 0)
>>> malformed_timespan.is_well_formed
False
\end{lstlisting}
\end{singlespacing}
%%% ABJADBOOK END %%%

\begin{comment}
<abjad>
templated_timespan = new(timespan, stop_offset=(5, 16))
print(format(templated_timespan))
</abjad>
\end{comment}

%%% ABJADBOOK START %%%
\begin{singlespacing}
\begin{lstlisting}
>>> templated_timespan = new(timespan, stop_offset=(5, 16))
>>> print(format(templated_timespan))
timespantools.Timespan(
    start_offset=durationtools.Offset(1, 4),
    stop_offset=durationtools.Offset(5, 16),
    )
\end{lstlisting}
\end{singlespacing}
%%% ABJADBOOK END %%%

\begin{comment}
<abjad>
annotated_timespan = timespantools.AnnotatedTimespan(
    start_offset=(1, 8),
    stop_offset=(7, 8),
    annotation='Any arbitrary object can act as an annotation.'
    )
annotated_timespan.annotation
</abjad>
\end{comment}

%%% ABJADBOOK START %%%
\begin{singlespacing}
\begin{lstlisting}
>>> annotated_timespan = timespantools.AnnotatedTimespan(
...     start_offset=(1, 8),
...     stop_offset=(7, 8),
...     annotation='Any arbitrary object can act as an annotation.'
...     )
>>> annotated_timespan.annotation
'Any arbitrary object can act as an annotation.'
\end{lstlisting}
\end{singlespacing}
%%% ABJADBOOK END %%%

\subsection{time relations}

- comparison

- time relations: intersection, congruency etc.

\begin{comment}
<abjad>
timespan_1 = timespantools.Timespan(0, 10)
timespan_2 = timespantools.Timespan(5, 15)
timespan_3 = timespantools.Timespan(10, 15)
</abjad>
\end{comment}

%%% ABJADBOOK START %%%
\begin{singlespacing}
\begin{lstlisting}
>>> timespan_1 = timespantools.Timespan(0, 10)
>>> timespan_2 = timespantools.Timespan(5, 15)
>>> timespan_3 = timespantools.Timespan(10, 15)
\end{lstlisting}
\end{singlespacing}
%%% ABJADBOOK END %%%

\begin{comment}
<abjad>
timespan_1.intersects_timespan(timespan_2)
timespan_1.intersects_timespan(timespan_3)
timespan_2.intersects_timespan(timespan_1)
timespan_2.intersects_timespan(timespan_3)
timespan_3.intersects_timespan(timespan_1)
timespan_3.intersects_timespan(timespan_2)
</abjad>
\end{comment}

%%% ABJADBOOK START %%%
\begin{singlespacing}
\begin{lstlisting}
>>> timespan_1.intersects_timespan(timespan_2)
True
\end{lstlisting}
\begin{lstlisting}
>>> timespan_1.intersects_timespan(timespan_3)
False
\end{lstlisting}
\begin{lstlisting}
>>> timespan_2.intersects_timespan(timespan_1)
True
\end{lstlisting}
\begin{lstlisting}
>>> timespan_2.intersects_timespan(timespan_3)
True
\end{lstlisting}
\begin{lstlisting}
>>> timespan_3.intersects_timespan(timespan_1)
False
\end{lstlisting}
\begin{lstlisting}
>>> timespan_3.intersects_timespan(timespan_2)
True
\end{lstlisting}
\end{singlespacing}
%%% ABJADBOOK END %%%

\begin{comment}
<abjad>
timespan_1.is_congruent_to_timespan(timespan_2)
timespan_1.is_congruent_to_timespan(timespan_1)
</abjad>
\end{comment}

%%% ABJADBOOK START %%%
\begin{singlespacing}
\begin{lstlisting}
>>> timespan_1.is_congruent_to_timespan(timespan_2)
False
\end{lstlisting}
\begin{lstlisting}
>>> timespan_1.is_congruent_to_timespan(timespan_1)
True
\end{lstlisting}
\end{singlespacing}
%%% ABJADBOOK END %%%

\begin{comment}
<abjad>
timespan_1.is_tangent_to_timespan(timespan_2)
timespan_1.is_tangent_to_timespan(timespan_3)
</abjad>
\end{comment}

%%% ABJADBOOK START %%%
\begin{singlespacing}
\begin{lstlisting}
>>> timespan_1.is_tangent_to_timespan(timespan_2)
False
\end{lstlisting}
\begin{lstlisting}
>>> timespan_1.is_tangent_to_timespan(timespan_3)
True
\end{lstlisting}
\end{singlespacing}
%%% ABJADBOOK END %%%

- operations

Consider the following three timespans again.

\begin{comment}
<abjad>
timespan_1 = timespantools.Timespan(0, 10)
timespan_2 = timespantools.Timespan(5, 15)
timespan_3 = timespantools.Timespan(10, 15)
</abjad>
\end{comment}

%%% ABJADBOOK START %%%
\begin{singlespacing}
\begin{lstlisting}
>>> timespan_1 = timespantools.Timespan(0, 10)
>>> timespan_2 = timespantools.Timespan(5, 15)
>>> timespan_3 = timespantools.Timespan(10, 15)
\end{lstlisting}
\end{singlespacing}
%%% ABJADBOOK END %%%

The logical AND of any two timespans can be computed.

\begin{comment}
<abjad>
timespan_1 & timespan_2
timespan_1 & timespan_3
timespan_2 & timespan_3
</abjad>
\end{comment}

%%% ABJADBOOK START %%%
\begin{singlespacing}
\begin{lstlisting}
>>> timespan_1 & timespan_2
TimespanInventory([Timespan(start_offset=Offset(5, 1), stop_offset=Offset(10, 1))])
\end{lstlisting}
\begin{lstlisting}
>>> timespan_1 & timespan_3
TimespanInventory([])
\end{lstlisting}
\begin{lstlisting}
>>> timespan_2 & timespan_3
TimespanInventory([Timespan(start_offset=Offset(10, 1), stop_offset=Offset(15, 1))])
\end{lstlisting}
\end{singlespacing}
%%% ABJADBOOK END %%%

The logical OR of any two timespans can be computed.

\begin{comment}
<abjad>
timespan_1 | timespan_2
timespan_1 | timespan_3
timespan_2 | timespan_3
</abjad>
\end{comment}

%%% ABJADBOOK START %%%
\begin{singlespacing}
\begin{lstlisting}
>>> timespan_1 | timespan_2
TimespanInventory([Timespan(start_offset=Offset(0, 1), stop_offset=Offset(15, 1))])
\end{lstlisting}
\begin{lstlisting}
>>> timespan_1 | timespan_3
TimespanInventory([Timespan(start_offset=Offset(0, 1), stop_offset=Offset(15, 1))])
\end{lstlisting}
\begin{lstlisting}
>>> timespan_2 | timespan_3
TimespanInventory([Timespan(start_offset=Offset(5, 1), stop_offset=Offset(15, 1))])
\end{lstlisting}
\end{singlespacing}
%%% ABJADBOOK END %%%

Timespan subtraction is another crucial operation.

\begin{comment}
<abjad>
timespan_1 = timespantools.Timespan(0, 15)
timespan_2 = timespantools.Timespan(5, 10)
timespan_3 = timespantools.Timespan(10, 20)
</abjad>
\end{comment}

%%% ABJADBOOK START %%%
\begin{singlespacing}
\begin{lstlisting}
>>> timespan_1 = timespantools.Timespan(0, 15)
>>> timespan_2 = timespantools.Timespan(5, 10)
>>> timespan_3 = timespantools.Timespan(10, 20)
\end{lstlisting}
\end{singlespacing}
%%% ABJADBOOK END %%%

\begin{comment}
<abjad>
print(format(timespan_1 - timespan_1))
print(format(timespan_1 - timespan_2))
print(format(timespan_1 - timespan_3))
print(format(timespan_2 - timespan_1))
print(format(timespan_2 - timespan_2))
print(format(timespan_2 - timespan_3))
print(format(timespan_3 - timespan_1))
print(format(timespan_3 - timespan_2))
print(format(timespan_3 - timespan_3))
</abjad>
\end{comment}

%%% ABJADBOOK START %%%
\begin{singlespacing}
\begin{lstlisting}
>>> print(format(timespan_1 - timespan_1))
timespantools.TimespanInventory(
    []
    )
\end{lstlisting}
\begin{lstlisting}
>>> print(format(timespan_1 - timespan_2))
timespantools.TimespanInventory(
    [
        timespantools.Timespan(
            start_offset=durationtools.Offset(0, 1),
            stop_offset=durationtools.Offset(5, 1),
            ),
        timespantools.Timespan(
            start_offset=durationtools.Offset(10, 1),
            stop_offset=durationtools.Offset(15, 1),
            ),
        ]
    )
\end{lstlisting}
\begin{lstlisting}
>>> print(format(timespan_1 - timespan_3))
timespantools.TimespanInventory(
    [
        timespantools.Timespan(
            start_offset=durationtools.Offset(0, 1),
            stop_offset=durationtools.Offset(10, 1),
            ),
        ]
    )
\end{lstlisting}
\begin{lstlisting}
>>> print(format(timespan_2 - timespan_1))
timespantools.TimespanInventory(
    []
    )
\end{lstlisting}
\begin{lstlisting}
>>> print(format(timespan_2 - timespan_2))
timespantools.TimespanInventory(
    []
    )
\end{lstlisting}
\begin{lstlisting}
>>> print(format(timespan_2 - timespan_3))
timespantools.TimespanInventory(
    [
        timespantools.Timespan(
            start_offset=durationtools.Offset(5, 1),
            stop_offset=durationtools.Offset(10, 1),
            ),
        ]
    )
\end{lstlisting}
\begin{lstlisting}
>>> print(format(timespan_3 - timespan_1))
timespantools.TimespanInventory(
    [
        timespantools.Timespan(
            start_offset=durationtools.Offset(15, 1),
            stop_offset=durationtools.Offset(20, 1),
            ),
        ]
    )
\end{lstlisting}
\begin{lstlisting}
>>> print(format(timespan_3 - timespan_2))
timespantools.TimespanInventory(
    [
        timespantools.Timespan(
            start_offset=durationtools.Offset(10, 1),
            stop_offset=durationtools.Offset(20, 1),
            ),
        ]
    )
\end{lstlisting}
\begin{lstlisting}
>>> print(format(timespan_3 - timespan_3))
timespantools.TimespanInventory(
    []
    )
\end{lstlisting}
\end{singlespacing}
%%% ABJADBOOK END %%%

\subsection{aggregate operations}

Timespans can be aggregated together in an instance of the TimespanInventory
class. In addition to the protocol defined for ordered collections, timespan
inventories provide a variety of other methods and properties for working
specifically with timespans.

\begin{comment}
<abjad>
timespan_inventory = timespantools.TimespanInventory([
    timespantools.Timespan(0, 16),
    timespantools.Timespan(5, 12),
    timespantools.Timespan(-2, 8),
    ])
timespan_inventory.timespan
timespan_inventory.duration
timespan_inventory.start_offset
timespan_inventory.stop_offset
timespan_inventory.append(timespantools.Timespan(15, 20))
timespan_inventory.sort()
timespan_inventory.duration
</abjad>
\end{comment}

%%% ABJADBOOK START %%%
\begin{singlespacing}
\begin{lstlisting}
>>> timespan_inventory = timespantools.TimespanInventory([
...     timespantools.Timespan(0, 16),
...     timespantools.Timespan(5, 12),
...     timespantools.Timespan(-2, 8),
...     ])
>>> timespan_inventory.timespan
Timespan(start_offset=Offset(-2, 1), stop_offset=Offset(16, 1))
\end{lstlisting}
\begin{lstlisting}
>>> timespan_inventory.duration
Duration(18, 1)
\end{lstlisting}
\begin{lstlisting}
>>> timespan_inventory.start_offset
Offset(-2, 1)
\end{lstlisting}
\begin{lstlisting}
>>> timespan_inventory.stop_offset
Offset(16, 1)
\end{lstlisting}
\begin{lstlisting}
>>> timespan_inventory.append(timespantools.Timespan(15, 20))
>>> timespan_inventory.sort()
>>> timespan_inventory.duration
Duration(22, 1)
\end{lstlisting}
\end{singlespacing}
%%% ABJADBOOK END %%%

- unioning, differencing and splitting

\begin{comment}
<abjad>
timespan_inventory = timespantools.TimespanInventory([
    timespantools.Timespan(0, 16),
    timespantools.Timespan(5, 12),
    timespantools.Timespan(-2, 8),
    ])
timespan = timespantools.Timespan(5, 10)
result = timespan_inventory & timespan
print(format(timespan_inventory))
</abjad>
\end{comment}

%%% ABJADBOOK START %%%
\begin{singlespacing}
\begin{lstlisting}
>>> timespan_inventory = timespantools.TimespanInventory([
...     timespantools.Timespan(0, 16),
...     timespantools.Timespan(5, 12),
...     timespantools.Timespan(-2, 8),
...     ])
>>> timespan = timespantools.Timespan(5, 10)
>>> result = timespan_inventory & timespan
>>> print(format(timespan_inventory))
timespantools.TimespanInventory(
    [
        timespantools.Timespan(
            start_offset=durationtools.Offset(5, 1),
            stop_offset=durationtools.Offset(8, 1),
            ),
        timespantools.Timespan(
            start_offset=durationtools.Offset(5, 1),
            stop_offset=durationtools.Offset(10, 1),
            ),
        timespantools.Timespan(
            start_offset=durationtools.Offset(5, 1),
            stop_offset=durationtools.Offset(10, 1),
            ),
        ]
    )
\end{lstlisting}
\end{singlespacing}
%%% ABJADBOOK END %%%

\begin{comment}
<abjad>
timespan_inventory = timespantools.TimespanInventory([
    timespantools.Timespan(0, 16),
    timespantools.Timespan(5, 12),
    timespantools.Timespan(-2, 8),
    ])
timespan = timespantools.Timespan(5, 10)
result = timespan_inventory - timespan
print(format(timespan_inventory))
</abjad>
\end{comment}

%%% ABJADBOOK START %%%
\begin{singlespacing}
\begin{lstlisting}
>>> timespan_inventory = timespantools.TimespanInventory([
...     timespantools.Timespan(0, 16),
...     timespantools.Timespan(5, 12),
...     timespantools.Timespan(-2, 8),
...     ])
>>> timespan = timespantools.Timespan(5, 10)
>>> result = timespan_inventory - timespan
>>> print(format(timespan_inventory))
timespantools.TimespanInventory(
    [
        timespantools.Timespan(
            start_offset=durationtools.Offset(-2, 1),
            stop_offset=durationtools.Offset(5, 1),
            ),
        timespantools.Timespan(
            start_offset=durationtools.Offset(0, 1),
            stop_offset=durationtools.Offset(5, 1),
            ),
        timespantools.Timespan(
            start_offset=durationtools.Offset(10, 1),
            stop_offset=durationtools.Offset(12, 1),
            ),
        timespantools.Timespan(
            start_offset=durationtools.Offset(10, 1),
            stop_offset=durationtools.Offset(16, 1),
            ),
        ]
    )
\end{lstlisting}
\end{singlespacing}
%%% ABJADBOOK END %%%

\begin{comment}
<abjad>
timespan_inventory = timespantools.TimespanInventory([
    timespantools.Timespan(0, 3),
    timespantools.Timespan(3, 6),
    timespantools.Timespan(6, 10),
    ])
left, right = timespan_inventory.split_at_offset(4)
print(format(left))
print(format(right))
</abjad>
\end{comment}

%%% ABJADBOOK START %%%
\begin{singlespacing}
\begin{lstlisting}
>>> timespan_inventory = timespantools.TimespanInventory([
...     timespantools.Timespan(0, 3),
...     timespantools.Timespan(3, 6),
...     timespantools.Timespan(6, 10),
...     ])
>>> left, right = timespan_inventory.split_at_offset(4)
>>> print(format(left))
timespantools.TimespanInventory(
    [
        timespantools.Timespan(
            start_offset=durationtools.Offset(0, 1),
            stop_offset=durationtools.Offset(3, 1),
            ),
        timespantools.Timespan(
            start_offset=durationtools.Offset(3, 1),
            stop_offset=durationtools.Offset(4, 1),
            ),
        ]
    )
\end{lstlisting}
\begin{lstlisting}
>>> print(format(right))
timespantools.TimespanInventory(
    [
        timespantools.Timespan(
            start_offset=durationtools.Offset(4, 1),
            stop_offset=durationtools.Offset(6, 1),
            ),
        timespantools.Timespan(
            start_offset=durationtools.Offset(6, 1),
            stop_offset=durationtools.Offset(10, 1),
            ),
        ]
    )
\end{lstlisting}
\end{singlespacing}
%%% ABJADBOOK END %%%

\begin{comment}
Timespan.split_at_offsets()
\end{comment}

- partitioning

\begin{comment}
<abjad>
timespan_inventory = timespantools.TimespanInventory([
    timespantools.Timespan(0, 10),
    timespantools.Timespan(5, 15),
    timespantools.Timespan(15, 20),
    timespantools.Timespan(25, 30),
    ])
</abjad>
\end{comment}

%%% ABJADBOOK START %%%
\begin{singlespacing}
\begin{lstlisting}
>>> timespan_inventory = timespantools.TimespanInventory([
...     timespantools.Timespan(0, 10),
...     timespantools.Timespan(5, 15),
...     timespantools.Timespan(15, 20),
...     timespantools.Timespan(25, 30),
...     ])
\end{lstlisting}
\end{singlespacing}
%%% ABJADBOOK END %%%

\begin{comment}
<abjad>
for inventory in timespan_inventory.partition():
    print(format(inventory))

</abjad>
\end{comment}

%%% ABJADBOOK START %%%
\begin{singlespacing}
\begin{lstlisting}
>>> for inventory in timespan_inventory.partition():
...     print(format(inventory))
...
timespantools.TimespanInventory(
    [
        timespantools.Timespan(
            start_offset=durationtools.Offset(0, 1),
            stop_offset=durationtools.Offset(10, 1),
            ),
        timespantools.Timespan(
            start_offset=durationtools.Offset(5, 1),
            stop_offset=durationtools.Offset(15, 1),
            ),
        ]
    )
timespantools.TimespanInventory(
    [
        timespantools.Timespan(
            start_offset=durationtools.Offset(15, 1),
            stop_offset=durationtools.Offset(20, 1),
            ),
        ]
    )
timespantools.TimespanInventory(
    [
        timespantools.Timespan(
            start_offset=durationtools.Offset(25, 1),
            stop_offset=durationtools.Offset(30, 1),
            ),
        ]
    )
\end{lstlisting}
\end{singlespacing}
%%% ABJADBOOK END %%%

\begin{comment}
<abjad>
for inventory in timespan_inventory.partition(include_tangent_timespans=True):
    print(format(inventory))

</abjad>
\end{comment}

%%% ABJADBOOK START %%%
\begin{singlespacing}
\begin{lstlisting}
>>> for inventory in timespan_inventory.partition(include_tangent_timespans=True):
...     print(format(inventory))
...
timespantools.TimespanInventory(
    [
        timespantools.Timespan(
            start_offset=durationtools.Offset(0, 1),
            stop_offset=durationtools.Offset(10, 1),
            ),
        timespantools.Timespan(
            start_offset=durationtools.Offset(5, 1),
            stop_offset=durationtools.Offset(15, 1),
            ),
        timespantools.Timespan(
            start_offset=durationtools.Offset(15, 1),
            stop_offset=durationtools.Offset(20, 1),
            ),
        ]
    )
timespantools.TimespanInventory(
    [
        timespantools.Timespan(
            start_offset=durationtools.Offset(25, 1),
            stop_offset=durationtools.Offset(30, 1),
            ),
        ]
    )
\end{lstlisting}
\end{singlespacing}
%%% ABJADBOOK END %%%

- multiplexing and demultiplexing

- resolution

- other operations

\begin{comment}
TimespanInventory.all_are_contiguous
TimespanInventory.all_are_nonoverlapping
TimespanInventory.clip_timespan_durations
TimespanInventory.count_offsets()
TimespanInventory.explode()
TimespanInventory.round_offsets()
\end{comment}

- timespan collection vs timespan inventory

Consort provides its own timespan collection class -- the TimespanCollection.
This class stores timespans internally not as a list, but in a balanced
"interval tree" datastructure which guarantees sorting and allows for highly
optimized lookups of timespans intersecting specific offsets. This class is
used at crucial points during Consort's interpretation stage simply for
purposes of speed, and should be considered an implementation detail. With
work, its internal datastructure will eventually be merged into Abjad's
TimespanInventory.

\section{performed and silent timespans}

\begin{comment}
<abjad>
performed_timespan = consort.PerformedTimespan(
    layer=1,
    minimum_duration=Duration(1, 8),
    music_specifier=consort.MusicSpecifier(),
    start_offset=Offset(1, 4),
    stop_offset=Offset(2, 1),
    voice_name='Violin 1 LH Voice',
    )
</abjad>
\end{comment}

%%% ABJADBOOK START %%%
\begin{singlespacing}
\begin{lstlisting}
>>> performed_timespan = consort.PerformedTimespan(
...     layer=1,
...     minimum_duration=Duration(1, 8),
...     music_specifier=consort.MusicSpecifier(),
...     start_offset=Offset(1, 4),
...     stop_offset=Offset(2, 1),
...     voice_name='Violin 1 LH Voice',
...     )
\end{lstlisting}
\end{singlespacing}
%%% ABJADBOOK END %%%

\begin{comment}
<abjad>
silent_timespan = consort.SilentTimespan(
    layer=2,
    start_offset=Offset(0, 1),
    stop_offset=Offset(1, 4),
    voice_name='Violin 1 LH Voice',
    )
</abjad>
\end{comment}

%%% ABJADBOOK START %%%
\begin{singlespacing}
\begin{lstlisting}
>>> silent_timespan = consort.SilentTimespan(
...     layer=2,
...     start_offset=Offset(0, 1),
...     stop_offset=Offset(1, 4),
...     voice_name='Violin 1 LH Voice',
...     )
\end{lstlisting}
\end{singlespacing}
%%% ABJADBOOK END %%%

\subsection{payloaded timespans}

- layer

- voice name

\subsection{performed timespans}

- forbid fusing

- forbid splitting

- minimum duration

- (additionally, music specifier: minimum phrase duration)

- divisions

- music

- music specifier

\section{timespan makers}

- timespan specifier

- independent vs dependent

- target timespans

- talea

- padding

\subsection{flooded timespan maker}

\begin{comment}
<abjad>
flooded_timespan_maker = consort.FloodedTimespanMaker()
print(format(flooded_timespan_maker))
</abjad>
\end{comment}

%%% ABJADBOOK START %%%
\begin{singlespacing}
\begin{lstlisting}
>>> flooded_timespan_maker = consort.FloodedTimespanMaker()
>>> print(format(flooded_timespan_maker))
consort.tools.FloodedTimespanMaker()
\end{lstlisting}
\end{singlespacing}
%%% ABJADBOOK END %%%

\subsection{talea timespan maker}

\begin{comment}
<abjad>
timespan_maker = consort.TaleaTimespanMaker(
    initial_silence_talea=rhythmmakertools.Talea(
        counts=(0, 4),
        denominator=16,
        )
    )
</abjad>

%%% ABJADBOOK START %%%
\begin{singlespacing}
\begin{lstlisting}
>>> timespan_maker = consort.TaleaTimespanMaker(
...     initial_silence_talea=rhythmmakertools.Talea(
...         counts=(0, 4),
...         denominator=16,
...         )
...     )
\end{lstlisting}
\end{singlespacing}
%%% ABJADBOOK END %%%

- taleas: playing, silence and initial silence

- groupings

- synchronization

- repeat and reflect

\subsection{dependent timespan maker}

\begin{comment}
<abjad>
dependent_timespan_maker = consort.DependentTimespanMaker(
    include_inner_starts=True,
    include_inner_stops=False,
    voice_names=(
        'Piano Upper Voice',
        'Piano Lower Voice',
        )
    )
</abjad>
\end{comment}

%%% ABJADBOOK START %%%
\begin{singlespacing}
\begin{lstlisting}
>>> dependent_timespan_maker = consort.DependentTimespanMaker(
...     include_inner_starts=True,
...     include_inner_stops=False,
...     voice_names=(
...         'Piano Upper Voice',
...         'Piano Lower Voice',
...         )
...     )
\end{lstlisting}
\end{singlespacing}
%%% ABJADBOOK END %%%

\section{rhythm makers}

\subsection{a factory for rhythmic material}

- divisions

\begin{comment}
<abjad>
divisions = [(3, 8), (4, 8), (3, 16), (4, 16), (5, 8), (2, 4)]
</abjad>
\end{comment}

%%% ABJADBOOK START %%%
\begin{singlespacing}
\begin{lstlisting}
>>> divisions = [(3, 8), (4, 8), (3, 16), (4, 16), (5, 8), (2, 4)]
\end{lstlisting}
\end{singlespacing}
%%% ABJADBOOK END %%%

- rhythm maker

\subsection{configuration}

- specifiers: tie, duration spelling, beam

\subsection{examples}

- specific rhythm makers

- NoteRhythmMaker

\begin{comment}
<abjad>
note_rhythm_maker = rhythmmakertools.NoteRhythmMaker(
    )
show(note_rhythm_maker, divisions=divisions)
</abjad>
\end{comment}

%%% ABJADBOOK START %%%
\begin{singlespacing}
\begin{lstlisting}
>>> note_rhythm_maker = rhythmmakertools.NoteRhythmMaker(
...     )
>>> show(note_rhythm_maker, divisions=divisions)
\end{lstlisting}
\includegraphics{assets/lilypond-d0f66021e4860194d32ee0fa226ed174.pdf}
\end{singlespacing}
%%% ABJADBOOK END %%%

- EvenDivisionsRhythmMaker

\begin{comment}
<abjad>
even_division_rhythm_maker = rhythmmakertools.EvenDivisionRhythmMaker(
    denominators=[8, 16, 4],
    )
show(even_division_rhythm_maker, divisions=divisions)
</abjad>
\end{comment}

%%% ABJADBOOK START %%%
\begin{singlespacing}
\begin{lstlisting}
>>> even_division_rhythm_maker = rhythmmakertools.EvenDivisionRhythmMaker(
...     denominators=[8, 16, 4],
...     )
>>> show(even_division_rhythm_maker, divisions=divisions)
\end{lstlisting}
\includegraphics{assets/lilypond-06c5f723997363adb1e3e9391d01892a.pdf}
\end{singlespacing}
%%% ABJADBOOK END %%%

- IncisedRhythmMaker

\begin{comment}
<abjad>
incised_rhythm_maker = rhythmmakertools.IncisedRhythmMaker(
    incise_specifier=rhythmmakertools.InciseSpecifier(
        prefix_counts=[0],
        suffix_talea=[-1],
        suffix_counts=[1],
        talea_denominator=16,
        ),
    )
show(incised_rhythm_maker, divisions=divisions)
</abjad>
\end{comment}

%%% ABJADBOOK START %%%
\begin{singlespacing}
\begin{lstlisting}
>>> incised_rhythm_maker = rhythmmakertools.IncisedRhythmMaker(
...     incise_specifier=rhythmmakertools.InciseSpecifier(
...         prefix_counts=[0],
...         suffix_talea=[-1],
...         suffix_counts=[1],
...         talea_denominator=16,
...         ),
...     )
>>> show(incised_rhythm_maker, divisions=divisions)
\end{lstlisting}
\includegraphics{assets/lilypond-f21215b14c3e76d687060a020bdca52a.pdf}
\end{singlespacing}
%%% ABJADBOOK END %%%

- TaleaRhythmMaker

\begin{comment}
<abjad>
talea_rhythm_maker = rhythmmakertools.TaleaRhythmMaker(
    talea=rhythmmakertools.Talea(
        counts=[1, 2, 3, 4],
        denominator=16,
        ),
    )
show(talea_rhythm_maker, divisions=divisions)
</abjad>
\end{comment}

%%% ABJADBOOK START %%%
\begin{singlespacing}
\begin{lstlisting}
>>> talea_rhythm_maker = rhythmmakertools.TaleaRhythmMaker(
...     talea=rhythmmakertools.Talea(
...         counts=[1, 2, 3, 4],
...         denominator=16,
...         ),
...     )
>>> show(talea_rhythm_maker, divisions=divisions)
\end{lstlisting}
\includegraphics{assets/lilypond-101e0079ec09025cbd79b4f02fb37c60.pdf}
\end{singlespacing}
%%% ABJADBOOK END %%%

\subsection{composite rhythm maker}

\begin{comment}
<abjad>
composite_rhythm_maker = consort.CompositeRhythmMaker(
    default=note_rhythm_maker,
    last=incised_rhythm_maker,
    first=even_division_rhythm_maker,
    )
</abjad>
\end{comment}

%%% ABJADBOOK START %%%
\begin{singlespacing}
\begin{lstlisting}
>>> composite_rhythm_maker = consort.CompositeRhythmMaker(
...     default=note_rhythm_maker,
...     last=incised_rhythm_maker,
...     first=even_division_rhythm_maker,
...     )
\end{lstlisting}
\end{singlespacing}
%%% ABJADBOOK END %%%

\section{meter finding and rewriting}

\subsection{describing meter}

- meters vs time signatures

- rhythm trees

\begin{comment}
<abjad>
three_four_meter = metertools.Meter((3, 4))
five_sixteen_meter = metertools.Meter((5, 16))
six_eight_meter = metertools.Meter((6, 8))
print(three_four_meter.pretty_rtm_format)
print(five_sixteen_meter.pretty_rtm_format)
print(six_eight_meter.pretty_rtm_format)
</abjad>
\end{comment}

%%% ABJADBOOK START %%%
\begin{singlespacing}
\begin{lstlisting}
>>> three_four_meter = metertools.Meter((3, 4))
>>> five_sixteen_meter = metertools.Meter((5, 16))
>>> six_eight_meter = metertools.Meter((6, 8))
>>> print(three_four_meter.pretty_rtm_format)
(3/4 (
	1/4
	1/4
	1/4))
\end{lstlisting}
\begin{lstlisting}
>>> print(five_sixteen_meter.pretty_rtm_format)
(5/16 (
	1/16
	1/16
	1/16
	1/16
	1/16))
\end{lstlisting}
\begin{lstlisting}
>>> print(six_eight_meter.pretty_rtm_format)
(6/8 (
	(3/8 (
		1/8
		1/8
		1/8))
	(3/8 (
		1/8
		1/8
		1/8))))
\end{lstlisting}
\end{singlespacing}
%%% ABJADBOOK END %%%

\subsection{finding meters}

\begin{comment}
<abjad>
permitted_meters = [metertools.Meter(_) for _ in [(3, 4), (4, 4), (5, 4)]]
offsets = [(0, 4), (4, 4), (8, 4), (12, 4), (16, 4)]
for x in metertools.Meter.fit_meters_to_expr(offsets, permitted_meters):
    print(x.implied_time_signature)

</abjad>
\end{comment}

%%% ABJADBOOK START %%%
\begin{singlespacing}
\begin{lstlisting}
>>> permitted_meters = [metertools.Meter(_) for _ in [(3, 4), (4, 4), (5, 4)]]
>>> offsets = [(0, 4), (4, 4), (8, 4), (12, 4), (16, 4)]
>>> for x in metertools.Meter.fit_meters_to_expr(offsets, permitted_meters):
...     print(x.implied_time_signature)
...
4/4
4/4
4/4
4/4
\end{lstlisting}
\end{singlespacing}
%%% ABJADBOOK END %%%

\begin{comment}
<abjad>
offsets = [(0, 4), (3, 4), (5, 4), (10, 4), (15, 4), (20, 4)]
for x in metertools.Meter.fit_meters_to_expr(offsets, permitted_meters):
    print(x.implied_time_signature)

</abjad>
\end{comment}

%%% ABJADBOOK START %%%
\begin{singlespacing}
\begin{lstlisting}
>>> offsets = [(0, 4), (3, 4), (5, 4), (10, 4), (15, 4), (20, 4)]
>>> for x in metertools.Meter.fit_meters_to_expr(offsets, permitted_meters):
...     print(x.implied_time_signature)
...
3/4
4/4
3/4
5/4
5/4
\end{lstlisting}
\end{singlespacing}
%%% ABJADBOOK END %%%

- metric accent kernels

- offset counters

- discard final silence

\subsection{rewriting meters}

- specific iteration techniques

- boundary depth

- dot count