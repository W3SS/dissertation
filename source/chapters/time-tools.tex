\chapter{Modeling time, rhythm and meter}

\begin{comment}
<abjad>[hide=true]
import consort
</abjad>
\end{comment}

Consort's implementation of a model of composition relies on a number of
different but interrelated models of musical time.

Dichotomies: outside and inside the score hierarchy, with or without regard to
notation, "coarse" versus "fine" or "phrase" versus "event", vertical or
horizontal, metered and unmetered, potentially simultaneous or strictly
contiguous.

Timespans provide a coarse model of musical time, both in and outside of score
hierarchy.

Notated rhythm provides a fine model of musical time, from within score
hierarchy.

Meter coordinates time and rhythm vertically across score hierarchy, and
bridges the coarse and fine stages of rhythmic interpretation.

Meter is generated as a by-product of phrase-level composition. It is not
specified by-hand during composition. This is not out of any desire to valorize
automaticism, but simply because lacking any other compelling reason to
generate a series of meters I felt the best way for myself would be to have
those meters derive from some sort of pre-existent structure in my
compositional process.

A discussion of these time models and their implications will clarify a later
analysis of the implementation of Consort's score interpretation
stage.

\section{timespans and timespan inventories}

\subsection{anatomy of a timespan}

\begin{comment}
<abjad>
timespan = timespantools.Timespan(
    start_offset=Offset(1, 4),
    stop_offset=Offset(3, 2),
    )
</abjad>
\end{comment}

\begin{comment}
<abjad>
timespan.start_offset
</abjad>
\end{comment}

\begin{comment}
<abjad>
timespan.stop_offset
</abjad>
\end{comment}

\begin{comment}
<abjad>
timespan.duration
</abjad>
\end{comment}

\begin{comment}
<abjad>
timespan.is_well_formed
</abjad>
\end{comment}

\begin{comment}
<abjad>
malformed_timespan = timespantools.Timespan(0, 0)
malformed_timespan.is_well_formed
</abjad>
\end{comment}

\begin{comment}
<abjad>
templated_timespan = new(timespan, stop_offset=(5, 16))
print(format(templated_timespan))
</abjad>
\end{comment}

\begin{comment}
<abjad>
annotated_timespan = timespantools.AnnotatedTimespan(
    start_offset=(1, 8),
    stop_offset=(7, 8),
    annotation='Any arbitrary object can act as an annotation.'
    )
annotated_timespan.annotation
</abjad>
\end{comment}

\subsection{time relations}

- comparison

- time relations: intersection, congruency etc.

\begin{comment}
<abjad>
timespan_1 = timespantools.Timespan(0, 10)
timespan_2 = timespantools.Timespan(5, 15)
timespan_3 = timespantools.Timespan(10, 15)
</abjad>
\end{comment}

\begin{comment}
<abjad>
timespan_1.intersects_timespan(timespan_2)
timespan_1.intersects_timespan(timespan_3)
timespan_2.intersects_timespan(timespan_1)
timespan_2.intersects_timespan(timespan_3)
timespan_3.intersects_timespan(timespan_1)
timespan_3.intersects_timespan(timespan_2)
</abjad>
\end{comment}

\begin{comment}
<abjad>
timespan_1.is_congruent_to_timespan(timespan_2)
timespan_1.is_congruent_to_timespan(timespan_1)
</abjad>
\end{comment}

\begin{comment}
<abjad>
timespan_1.is_tangent_to_timespan(timespan_2)
timespan_1.is_tangent_to_timespan(timespan_3)
</abjad>
\end{comment}

- operations

Consider the following three timespans again.

\begin{comment}
<abjad>
timespan_1 = timespantools.Timespan(0, 10)
timespan_2 = timespantools.Timespan(5, 15)
timespan_3 = timespantools.Timespan(10, 15)
</abjad>
\end{comment}

The logical AND of any two timespans can be computed.

\begin{comment}
<abjad>
timespan_1 & timespan_2
timespan_1 & timespan_3
timespan_2 & timespan_3
</abjad>
\end{comment}

The logical OR of any two timespans can be computed.

\begin{comment}
<abjad>
timespan_1 | timespan_2
timespan_1 | timespan_3
timespan_2 | timespan_3
</abjad>
\end{comment}

Timespan subtraction is another crucial operation.

\begin{comment}
<abjad>
timespan_1 = timespantools.Timespan(0, 15)
timespan_2 = timespantools.Timespan(5, 10)
timespan_3 = timespantools.Timespan(10, 20)
</abjad>
\end{comment}

\begin{comment}
<abjad>
print(format(timespan_1 - timespan_1))
print(format(timespan_1 - timespan_2))
print(format(timespan_1 - timespan_3))
print(format(timespan_2 - timespan_1))
print(format(timespan_2 - timespan_2))
print(format(timespan_2 - timespan_3))
print(format(timespan_3 - timespan_1))
print(format(timespan_3 - timespan_2))
print(format(timespan_3 - timespan_3))
</abjad>
\end{comment}

\subsection{aggregate operations}

Timespans can be aggregated together in an instance of the TimespanInventory
class. In addition to the protocol defined for ordered collections, timespan
inventories provide a variety of other methods and properties for working
specifically with timespans.

\begin{comment}
<abjad>
timespan_inventory = timespantools.TimespanInventory([
    timespantools.Timespan(0, 16),
    timespantools.Timespan(5, 12),
    timespantools.Timespan(-2, 8),
    ])
timespan_inventory.timespan
timespan_inventory.duration
timespan_inventory.start_offset
timespan_inventory.stop_offset
timespan_inventory.append(timespantools.Timespan(15, 20))
timespan_inventory.sort()
timespan_inventory.duration
</abjad>
\end{comment}

- unioning, differencing and splitting

\begin{comment}
<abjad>
timespan_inventory = timespantools.TimespanInventory([
    timespantools.Timespan(0, 16),
    timespantools.Timespan(5, 12),
    timespantools.Timespan(-2, 8),
    ])
timespan = timespantools.Timespan(5, 10)
result = timespan_inventory & timespan
print(format(timespan_inventory))
</abjad>
\end{comment}

\begin{comment}
<abjad>
timespan_inventory = timespantools.TimespanInventory([
    timespantools.Timespan(0, 16),
    timespantools.Timespan(5, 12),
    timespantools.Timespan(-2, 8),
    ])
timespan = timespantools.Timespan(5, 10)
result = timespan_inventory - timespan
print(format(timespan_inventory))
</abjad>
\end{comment}

\begin{comment}
<abjad>
timespan_inventory = timespantools.TimespanInventory([
    timespantools.Timespan(0, 3),
    timespantools.Timespan(3, 6),
    timespantools.Timespan(6, 10),
    ])
left, right = timespan_inventory.split_at_offset(4)
print(format(left))
print(format(right))
</abjad>
\end{comment}

\begin{comment}
Timespan.split_at_offsets()
\end{comment}

- partitioning

\begin{comment}
<abjad>
timespan_inventory = timespantools.TimespanInventory([
    timespantools.Timespan(0, 10),
    timespantools.Timespan(5, 15),
    timespantools.Timespan(15, 20),
    timespantools.Timespan(25, 30),
    ])
</abjad>
\end{comment}

\begin{comment}
<abjad>
for inventory in timespan_inventory.partition():
    print(format(inventory))

</abjad>
\end{comment}

\begin{comment}
<abjad>
for inventory in timespan_inventory.partition(include_tangent_timespans=True):
    print(format(inventory))

</abjad>
\end{comment}

- multiplexing and demultiplexing

- resolution

- other operations

\begin{comment}
TimespanInventory.all_are_contiguous
TimespanInventory.all_are_nonoverlapping
TimespanInventory.clip_timespan_durations
TimespanInventory.count_offsets()
TimespanInventory.explode()
TimespanInventory.round_offsets()
\end{comment}

- timespan collection vs timespan inventory

Consort provides its own timespan collection class -- the TimespanCollection.
This class stores timespans internally not as a list, but in a balanced
"interval tree" datastructure which guarantees sorting and allows for highly
optimized lookups of timespans intersecting specific offsets. This class is
used at crucial points during Consort's interpretation stage simply for
purposes of speed, and should be considered an implementation detail. With
work, its internal datastructure will eventually be merged into Abjad's
TimespanInventory.

\section{performed and silent timespans}

\begin{comment}
<abjad>
performed_timespan = consort.PerformedTimespan(
    layer=1,
    minimum_duration=Duration(1, 8),
    music_specifier=consort.MusicSpecifier(),
    start_offset=Offset(1, 4),
    stop_offset=Offset(2, 1),
    voice_name='Violin 1 LH Voice',
    )
</abjad>
\end{comment}

\begin{comment}
<abjad>
silent_timespan = consort.SilentTimespan(
    layer=2,
    start_offset=Offset(0, 1),
    stop_offset=Offset(1, 4),
    voice_name='Violin 1 LH Voice',
    )
</abjad>
\end{comment}

\subsection{payloaded timespans}

- layer

- voice name

\subsection{performed timespans}

- forbid fusing

- forbid splitting

- minimum duration

- (additionally, music specifier: minimum phrase duration)

- divisions

- music

- music specifier

\section{timespan makers}

- timespan specifier

- independent vs dependent

- target timespans

- talea

- padding

\subsection{flooded timespan maker}

\begin{comment}
<abjad>
flooded_timespan_maker = consort.FloodedTimespanMaker()
print(format(flooded_timespan_maker))
</abjad>
\end{comment}

\subsection{talea timespan maker}

\begin{comment}
<abjad>
timespan_maker = consort.TaleaTimespanMaker(
    initial_silence_talea=rhythmmakertools.Talea(
        counts=(0, 4),
        denominator=16,
        )
    )
</abjad>

- taleas: playing, silence and initial silence

- groupings

- synchronization

- repeat and reflect

\subsection{dependent timespan maker}

\begin{comment}
<abjad>
dependent_timespan_maker = consort.DependentTimespanMaker(
    include_inner_starts=True,
    include_inner_stops=False,
    voice_names=(
        'Piano Upper Voice',
        'Piano Lower Voice',
        )
    )
</abjad>
\end{comment}

\section{rhythm makers}

\subsection{a factory for rhythmic material}

- divisions

\begin{comment}
<abjad>
divisions = [(3, 8), (4, 8), (3, 16), (4, 16), (5, 8), (2, 4)]
</abjad>
\end{comment}

- rhythm maker

\subsection{configuration}

- specifiers: tie, duration spelling, beam

\subsection{examples}

- specific rhythm makers

- NoteRhythmMaker

\begin{comment}
<abjad>
note_rhythm_maker = rhythmmakertools.NoteRhythmMaker(
    )
show(note_rhythm_maker, divisions=divisions)
</abjad>
\end{comment}

- EvenDivisionsRhythmMaker

\begin{comment}
<abjad>
even_division_rhythm_maker = rhythmmakertools.EvenDivisionRhythmMaker(
    denominators=[8, 16, 4],
    )
show(even_division_rhythm_maker, divisions=divisions)
</abjad>
\end{comment}

- IncisedRhythmMaker

\begin{comment}
<abjad>
incised_rhythm_maker = rhythmmakertools.IncisedRhythmMaker(
    incise_specifier=rhythmmakertools.InciseSpecifier(
        prefix_counts=[0],
        suffix_talea=[-1],
        suffix_counts=[1],
        talea_denominator=16,
        ),
    )
show(incised_rhythm_maker, divisions=divisions)
</abjad>
\end{comment}

- TaleaRhythmMaker

\begin{comment}
<abjad>
talea_rhythm_maker = rhythmmakertools.TaleaRhythmMaker(
    talea=rhythmmakertools.Talea(
        counts=[1, 2, 3, 4],
        denominator=16,
        ),
    )
show(talea_rhythm_maker, divisions=divisions)
</abjad>
\end{comment}

\subsection{composite rhythm maker}

\begin{comment}
<abjad>
composite_rhythm_maker = consort.CompositeRhythmMaker(
    default=note_rhythm_maker,
    last=incised_rhythm_maker,
    first=even_division_rhythm_maker,
    )
</abjad>
\end{comment}

\section{meter finding and rewriting}

\subsection{describing meter}

- meters vs time signatures

- rhythm trees

\begin{comment}
<abjad>
three_four_meter = metertools.Meter((3, 4))
five_sixteen_meter = metertools.Meter((5, 16))
six_eight_meter = metertools.Meter((6, 8))
print(three_four_meter.pretty_rtm_format)
print(five_sixteen_meter.pretty_rtm_format)
print(six_eight_meter.pretty_rtm_format)
</abjad>
\end{comment}

\subsection{finding meters}

\begin{comment}
<abjad>
permitted_meters = [metertools.Meter(_) for _ in [(3, 4), (4, 4), (5, 4)]]
offsets = [(0, 4), (4, 4), (8, 4), (12, 4), (16, 4)]
for x in metertools.Meter.fit_meters_to_expr(offsets, permitted_meters):
    print(x.implied_time_signature)

</abjad>
\end{comment}

\begin{comment}
<abjad>
offsets = [(0, 4), (3, 4), (5, 4), (10, 4), (15, 4), (20, 4)]
for x in metertools.Meter.fit_meters_to_expr(offsets, permitted_meters):
    print(x.implied_time_signature)

</abjad>
\end{comment}

- metric accent kernels

- offset counters

- discard final silence

\subsection{rewriting meters}

- specific iteration techniques

- boundary depth

- dot count